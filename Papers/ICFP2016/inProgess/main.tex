%-----------------------------------------------------------------------------
%
%               Template for sigplanconf LaTeX Class
%
% Name:         sigplanconf-template.tex
%
% Purpose:      A template for sigplanconf.cls, which is a LaTeX 2e class
%               file for SIGPLAN conference proceedings.
%
% Guide:        Refer to "Author's Guide to the ACM SIGPLAN Class,"
%               sigplanconf-guide.pdf
%
% Author:       Paul C. Anagnostopoulos
%               Windfall Software
%               978 371-2316
%               paul@windfall.com
%
% Created:      15 February 2005
%
%-----------------------------------------------------------------------------


\documentclass[preprint]{sigplanconf}

% The following \documentclass options may be useful:

% preprint      Remove this option only once the paper is in final form.
% 10pt          To set in 10-point type instead of 9-point.
% 11pt          To set in 11-point type instead of 9-point.
% numbers       To obtain numeric citation style instead of author/year.

\usepackage{amsmath}
\usepackage{listings}
\usepackage{xspace}
\newcommand{\ruzica}[1]{\textcolor{comment_color}{\textsf{RP}: #1}}
\newcommand{\markk}[1]{\textcolor{comment_color}{\textsf{MS}: #1}}
\newcommand{\alex}[1]{\textcolor{comment_color}{\textsf{AR}: #1}}


\def\ourTool/{Ives}
\def\lhask/{LiquidHaskell}

\newcommand{\codeinline}[1]{\lstinline[basicstyle=\small]{#1}}

\usepackage{tikz}



\newcommand{\cL}{{\cal L}}

\begin{document}

\special{papersize=8.5in,11in}
\setlength{\pdfpageheight}{\paperheight}
\setlength{\pdfpagewidth}{\paperwidth}

\conferenceinfo{CONF 'yy}{Month d--d, 20yy, City, ST, Country}
\copyrightyear{20yy}
\copyrightdata{978-1-nnnn-nnnn-n/yy/mm}
\copyrightdoi{nnnnnnn.nnnnnnn}

% Uncomment the publication rights you want to use.
%\publicationrights{transferred}
%\publicationrights{licensed}     % this is the default
%\publicationrights{author-pays}

\titlebanner{banner above paper title}        % These are ignored unless
\preprintfooter{short description of paper}   % 'preprint' option specified.

\title{Natural Program Synthesis from Examples in Haskell}

\authorinfo{Name1}
           {Affiliation1}
           {Email1}
\authorinfo{Name2\and Name3}
           {Affiliation2/3}
           {Email2/3}

\maketitle

\begin{abstract}
  We present a method for programming-by-example in a language that supports a full range of higher-order functions.
  This modular synthesis algorithm seperates higher order and first order function synthesis.
  Given a set of input-output examples, our method first selects the best fitting, potential higher-order functions.
  Combined with a simple, type-directed first order synthesis procedure, we can generate full, easy to read, programs.

  We use a \textit{dismantling} technique to infer refinement types over candidate higher-order functions.
  Then, we infer a set of potential refinement types over the input-output examples.
  Merging these two sets gives a ranked list of suitable higher-order functions.
  Using refinement types frees us from the need for restrictive templating, allowing us to use full libraries in our synthesis procedure.
  To complete the synthesis, we use a best-first enumerative search based on a weighted subtyping to remove the need for subexample generation, which would rely on further templating.

  We implement this approach in \ourTool/ against a set of synthesis examples including lists, trees, maps, and specialized musical score data structures.
  The synthesized programs operate over datastructures and draw functions from various real world Haskell packages.
  This evaluation demonstrates the scalability and versatility of this approach.
\end{abstract}

\category{CR-number}{subcategory}{third-level}

% general terms are not compulsory anymore,
% you may leave them out
\terms
term1, term2

\keywords
keyword1, keyword2




\section{Introduction}
\label{intro}

Program synthesis is an active research direction~\cite{DBLP:journals/toplas/MannaW80, DBLP:journals/cacm/GulwaniHS12,DBLP:conf/icfp/Bodik15, DBLP:conf/pldi/KuncakMPS10, DBLP:conf/aplas/Solar-Lezama09, DBLP:conf/pldi/SrivastavaGCF11} that aims to automatically derive code from a given specification.
This code is correct by construction and ideally would make a programmer more productive.
Still, writing a complete specification of an entire program is often a more complex task than writing the corresponding code, even for very simple programs.

Programming by example~\cite{cypher93,lieberman01,synasc12} is a promising research direction that enables easy manipulation of data even for non-programmers~\cite{GulwaniHS12}.
Recent work in this area has focused on manipulating fundamental data types such as strings~\cite{flashFillPOPL,vldb12,icml13}, lists~\cite{FeserCD15,Osera:2015} and numbers~\cite{cav12}.
The success and impact of this line of work can be estimated from the fact that some of this technology ships as part of the popular FlashFill feature in Excel 2013~\cite{flashFillPOPL}.

Instead of writing code, the user provides a list of relevant input/output examples and the synthesis tool automatically generates a program that fits.
In this way, the examples can be seen as an easily readable and understandable specification.
However, the code that is produced by these tools is rarely as simple as the specification.
%In~\cite{Osera:2016}, the close connection between refinement types and examples is expounded through the lens of proof theory.
%Other works have explored further ramifications of the theory behind programming by example\cite{Osera:2015, GulwaniHS12, synasc12}.
%These theoretical foundations give us the power and direction to begin to make programming by example a mainstream feature of fully featured languages.

Program synthesis does not inherently address the problem of code readability, often resulting in tools that produce something closer to an executable than the simple and stylistic code a human might write.
Just as with code users write themselves, the ability to reuse and edit synthesized code is an integral part of the programming process.
Thus, we define a notion of \textit{natural} code and define the problem of natural program synthesis.

Take the simple task of synthesizing a list flattening function from examples.
Synthesis approaches that use only the primitive recursive operators~\cite{Osera:2015,FeserCD15}, would find a function similar to \codeinline{solution1} in Listing \ref{natSyn} (modulo language syntax).
Our tool instead focuses on utilizing common library functions, as seen in \codeinline{solution2}.
To quantify our goal of simplicity, we define a notion of \textit{naturalness} of code in Section \ref{sec:naturalness}.
Intuitively, \codeinline{solution2} should have a higher naturalness score than \codeinline{solution1}.


\begin{figure}
  \begin{lstlisting}[caption=Low-level synthesis vs. Natural synthesis,label=natSyn]
    exs = ([[1,2],[3,4]], [1,2,3,4])

    solution1 xs =
      (\c n -> foldr
          (\x b -> foldr c b (id x))
          n xs) (:) []

    solution2 xs = concatMap id xs
    \end{lstlisting}
\end{figure}

\noindent In fact, \codeinline{solution1} is just an application of the GHC~\cite{ghc} implementation of concatMap.
To synthesize this solution using only the core higher order functions is certainly motivating, however a user would likely prefer to see a single function, like \codeinline{concatMap}, which exists in a standard library, when using synthesis with the goal of writing their own code.

We choose to focus on functional programming, where a core part of the experience is writing higher order functions~\cite{Lipovaca:2011}.
Library authors often provide users with a number of domain specific higher order functions to enable programmers to more easily write their programs.
Since users write higher order functions with a deep understanding of the domain, using them in synthesis produces code that is more idiomatic and easier to understand then using only the core higher order functions.

In order to facilitate synthesis over user defined higher order functions, we run the synthesis algorithm in two stages.
The first offline preprocessing stage infers rules about how user defined, or imported higher order functions behave over their input and output types.
We encode these rules using refinement types in Haskell with the \lhask/ tool~\cite{DBLP:conf/haskell/VazouSJ14}.
By using all the information available from the examples, we also infer refinement types on the example set. 
This is used during the online synthesis stage, along with a type match ranking algorithm to efficiently prune and navigate the search space of solution programs.


% fewer examples are needed when leveraging user functions
Although programming by example is an easy entry point for novice users, one of the drawbacks can be the tedious nature of the specification.
For a user, writing out a sufficient number of examples for the synthesis tool to find a solution may involve
  specifying seemingly obvious examples such as \codeinline{[]->[]} in order to cover base cases of recursion.
However, much of this domain specific knowledge is often encoded by the user defined functions, data types and library imports.
By focusing our synthesis procedures on this space, we can reduce the number of required edge case examples and allow users to only provide the important examples.

In Listing \ref{natSyn}, we have already seen the potential to synthesize natural solutions to programming by example queries.
However, a synthesis engine should also be able to synthesize novel (and sometimes unexpected) solutions to problems.
Since the stated goal is to find simple programs that a human might write, this raises the question if finding ``natural'' and and novel programs are at odds with each other.

Our evaluation section shows that this is not the case.
For example, natural synthesis of the Boolean ``or'' function finds two functions, \codeinline{any (id)} and \codeinline{foldr1 (max)}.
The \codeinline{any (id)} solution would expected by Haskell programmer, where \codeinline{any:: (a-> Bool) -> [a] -> Bool)} is a built-in function to Haskell that returns \codeinline{True} if any element of a list satisfies the predicate function.
The more novel solution returned by our natural synthesis is \codeinline{foldr1 (max)}, where \codeinline{max :: Ord a => a -> a -> a} will return the maximum element of the two inputs.
By folding over the list, this solution program exploits the \codeinline{Enum} property of the Boolean type in Haskell in a way that provides insight into some core Haskell functions and types.

We implement this approach in a tool, \ourTool/, that support the real Haskell language, including its native types and any user defined data types.
\ourTool/ synthesizes programs from very few examples to make the specification as simple as possible.
\ourTool/ places a high weight on natural code during the synthesis procedure.
It can also reuse user-defined and library functions in the synthesis procedure, thereby generating code that is both natural to the specific domain.
\ourTool/ is used as a standalone tool as many previous works have functioned, but because it handles proper Haskell, could also be integrated into an IDE.

In summary, we present the following contributions:

\begin{enumerate}
\item A definition of naturalness to describe the complexity of functional code, and a definition of the problem of natural program synthesis.
\item A weighted, type directed, enumeration strategy that generates programs of a high naturalness score for programs that also satisfy user provided examples. 
\item An exploitation of an implicit assumption in existing programming by example work to shrink the search space using refinement types.
\item An evaluation of our implementation, \ourTool/, in Haskell to synthesize programs that utilize Haskell libraries. The tool supports native and user defined data types to produce real Haskell code. The benchmarks show our tool can efficiently generate a wide variety of code that mixes functions from multiple sources.
\end{enumerate}


\section{Motivating Examples} 
\label{examples}

With next few illustrative examples we show functionality of \ourTool/.
It takes as input a list of examples and automatically generates code
corresponding to these examples. The tool is embedded in Haskell, so examples use regular Haskell syntax along with a type annotation.



\subsection{Synthesis with the standard library}
\label{sec:exampleBasic}

%introduction to us using stutter- from the ground up
We start with a \codeinline{stutter} program. It is one of canonical 
examples used by similar tools~\cite{Osera:2015}. The \codeinline{stutter} program takes as input a list of elements and duplicates each element 
of the list.
 
Following the programming-by-example principles, the user 
provide a list of examples as a list of input/output pairs to the special variable \codeinline{exs}. These examples should be
illustrative enough to convey her intentions:

\begin{lstlisting}
exs :: [([Int], [Int])]
exs = [([1, 2, 3], [1, 1, 2, 2, 3, 3])]
\end{lstlisting}

Invoking \ourTool/ on this example starts to construct programs  satisfying that when they are applied to the list $[1, 2, 3]$ the result 
is $[1, 1, 2, 2, 3, 3]$.
Using a type-directed enumeration, we look for programs fitting the given example type, in this case \codeinline{[Int]->[Int]}.
We first search for first order functions of type \codeinline{[Int]->[Int]}.
In addition we look for generalizations of the given type, such as \codeinline{[a]->[a]}, or \codeinline{Num a => [a]->[a]}.
There are various functions of this type, such as \codeinline{cycle} or \codeinline{tail}, but they do not satisfy the given example.
Once we have exhausted this class of functions, we begin searching for chains of application.
In particular, we focus on higher order functions because they are widely use in Haskell programs. Additionally, their use in program synthesis 
in general has not been studied extensively as for first-order functions.

For the given example a higher order function of interest is present in the Haskell the standard library; \codeinline{concatMap :: (a -> [b]) -> [a] -> [b]} which applies a function over a list and concatenates the result.
With \codeinline{concatMap} as a candidate higher order function, we focus
 our search for its first argument, which is a first order function of type \codeinline{a -> [b]}.
Since our algorithm does not find a useful function of such a type, we continue searching for generalizations of that type.
This way we find the function \codeinline{replicate n :: Int -> a -> [a]} in the standard library, which will replicate an item n times into a list.
This function only needs to be specialized to our examples by applying an initial value.

In the end, \ourTool/ returns the program \codeinline{concatMap (replicate 2)}. It only required a single example to synthesize this code.

However, in the above example we considered only functions from the standard Haskell library. Let us now extend the program with a user defined function \codeinline{dupl} which duplicates an element:
\begin{lstlisting}
dupl :: a -> [a]
dupl x = [x,x]
\end{lstlisting}

Clearly, there is now a second possible solution to the synthesis problem.
Rather than waiting to find all solutions, \ourTool/ returns solutions to the user as they are found, and then further proceeds with rest of the search.
In this example, \ourTool/ returns the solution \codeinline{concatMap dupl} as the first solution after X seconds, it continues searching and returns \codeinline{concatMap (replicate 2)} after X+Y seconds.

The \codeinline{stutter} program was also synthesized in MYTH~\cite{Osera:2015}. However, their focus was on lists as inductively defined data types. Instead we 
are focusing on using the built in representation of a list in Haskell.

\ourTool/ returns several solutions satisfying the given examples. In
 order to return better solutions in the beginning of the search, we use a ranking system. By ``better'' solution we mean solutions that are more intuitive, simpler, and more relevant to the user. Formally, we introduce a new complexity measure called \textit{naturalness} (defined in 
Sec.~\ref{sec:naturalness} and our goal is to find the most natural solution. We use a ranking system based on the complexity of the 
generated programs. The ranking system helps to guide our search.
Naturalness will also help us avoid the difficult to read solutions as those often seen in other programming-by-example systems.

Our ultimate goal is to integrate the programming-by-example paradigm 
into mainstream programming environments, thus the generated code must be readable and editable.
The code should also support a real language and use native data types to the language.


\subsection{Optimizing with types}
\label{sec:exampleFilter}

A na\"ive approach to the synthesis would be to search 
for every function of the type given by examples. However, this would be
a highly non-efficient algorithm.
Instead we use information embedded in the examples, to infer refinement types which help us to prune the search space.
Consider the following example:
\begin{lstlisting}
exs :: [([Int], [Int])]
exs = [([1, 2, 3], [1,3])]
\end{lstlisting}
The goal here is to synthesize a function that drops the odd numbers from a list. Based on only this single example, \ourTool/ finds a simple solution program \codeinline{filter (odd)}.

Rather then starting a search over all functions of type \codeinline{[Int] -> [Int]}, we can first infer a refinement 
type~\cite{DBLP:conf/icfp/VazouSJVJ14}.
This refinement type will specify that the size of the input list cannot
be shorter than the size of the output list.
\ourTool/ then searches for a higher order function consistent with this refinement type.
For simplicity, let us restrict our library to only following two
higher order functions \codeinline{map} and \codeinline{filter}.
\codeinline{map} has assigned a refinement type that specifies the input and output lengths are equal - thus it is pruned from the search since this observation is not consistent with the example's refinement type.
\codeinline{filter} has assigned a refinement type that specifies the input length is greater than or equal to the output length.
This observation is consistent with the examples, and is thus used to synthesize the solution \codeinline{filter (odd)}, using the methods described in previous example.


\subsection{Synthesis with user defined values}


Working on a set of user defined code is also a critical task \ourTool/ supports. 
In the next example the user has provided a binary tree data structure and a higher order function to map over it. We show the synthesis of the exceedingly (for the sake of brevity) simple program \codeinline{mapBTree not}.
Doing such synthesis requires automatic reasoning about not only the user defined polymorphic data type, but also the higher order function they have defined over it.

\begin{lstlisting}
data BTree a = Nil |
               Branch (BTree a) a (BTree a)

mapBTree :: (a -> a) -> BTree a -> BTree 
mapBTree f Nil = Nil
mapBTree f (Branch b1 v b2) = 
  Branch (mapBTree f b1) (f v) (mapBTree f b2)

exs :: [(BTree Bool, Tree Bool)]
exs = [(Branch Nil True Nil,
       Branch Nil False Nil)]
\end{lstlisting}

It may seem that if a user can write a higher order functions over custom data structures, they would not have a need to synthesize such functions.
However, imagine the case of a user importing libraries.
Haskell's module system and large repository of libraries like Hackage and Stackage are an indispensable part of the language\cite{hackage,stackage}.
Often, a user is importing a library that is large, unfamiliar, and/or poorly documented.
Using \ourTool/, the user no longer needs an intimate knowledge of the library to makes use of the functions and datatypes, and can instead synthesize functions from examples.


\subsection{Synthesis with a DSL}

As an example, we show code to transpose a music value from the Euterpea DSL (domain specific library) for music\cite{euterpea}.
Among other things, Euterpea defines a tree-like datatype called \codeinline{Music} and various functions for manipulating these types.
The user only needs to express the basic datatype as examples, and \ourTool/ can synthesize the \codeinline{solution} program.
The solution utilizes the functions from Euterpea; \codeinline{mMap} for mapping over music values, and \codeinline{(trans::Int -> Music Pitch->Music Pitch)} to transpose a Music Pitch by a value.
This again requires automated reasoning about the properties of the library's data types and higher order functions.
Because we have synthesized a natural looking program, the user does not need to understand details of the library's function and data structures to be able to immediately gain an intuition about how the solution program works.

\begin{lstlisting}
import Euterpea

exs :: [(Music Pitch, Music Pitch)]
exs = [
  (Prim (Note qn (C,4)):+:Prim (Note qn (D,4)),
   Prim (Note qn (D,4)):+:Prim (Note qn (E,4))) ]
        
solution = mMap (trans 2)
\end{lstlisting}



\section{Problem Formulation}
\label{problem}

\subsection{Naturalness}

The goal in natural synthesis is synthesize programs that meet a specification with the simplest and most readable code.
For programming by example, the specification is a set, $Ex$, of pairs of input and output $(i,o)$.
As a formalization of simplicity and readability, we present a definition of naturalness.

The standard of measuring complexity (the inverse of naturalness) is cyclomatic complexity\cite{cyclo}.
Cyclomatic complexity is function, $\mathcal{CC}$on the control flow graph of a section of code, $C$.
The function is $\mathcal{CC}(C) = E − N + 2(P)$, where $E$, $N$, and $P$ are the edges, nodes and connected components respectively.
This function measures "the number of linearly independent paths" through a program, a crucial part of manipulating stateful variables.
However a this is not well-suited for pure functional languages that use branching in a more functional way.
For example, given two programs to compute absolute value, the "if" solution should be more natural, but using cycolmatic complexity verbatim means
\codeinline{f x = if x>=0 then x else x*-1} is more complex
\codeinline{f x = x * (fromEnum (x<0) * (-1) + fromEnum (x>0))}

Instead, we define naturalness to be the number of nodes in the abstract syntax tree.
This approach integrates cyclomatic complexity, because branching is still encoded as a measure of complexity.

%Our definition of naturalness matches well many techniques valued in functional community.
Currying fits under this definition of naturalness.
\codeinline{f = (1+)} is more natural than \codeinline{f x= 1 + x}.
This also favors predefined functions over generating new lambda terms.
Taking the code from Listing \ref{natSyn}, $\mathcal{N}($solution1$)$ has a naturalness score of 1/8, while $\mathcal{N}($solution2$)$ has a score of 1/3.

The problem is then to find an expression e, such that $\forall (i,o) \in Ex, e (i) = o \land max(\mathcal{N}(e))$.


\subsection{Enumeration}
In order to solve the above problem, we will immediately restrict our search space to exclude generated lambda terms, as such terms will generally induce a very low naturalness score and explode the search space.
Our synthesis approach will then only be able to solve synthesis problems when a solution exists that only draws from a finite set of predefined expressions.

Let $E$ be the finite set of expressions exposed to the top-level module from user code, imported libraries, and the core library.
Our search space will be the set of permutations of well-typed applications of elements of $E$.

To determine if an expression is well-typed, let $T$ be the set of types (both inferred and explicitly declared) exposed to the top-level module from user code, imported libraries, and the core library.
A type environment $\Gamma$, is the set $\{e1 : \tau_1,\ ...,\ e_n : \tau_n\}$, where $e_{i} \in E$ and is of type $\tau_i \in T$.
The set of well-typed expressions we consider, $\mathcal{G}$, is the infinite set $\{e1 : \tau_1,\ ...,\ e_n : \tau_n\}$, where $e_i : \tau_i$ follows the usual rules of application for constructing well-typed expressions from $\Gamma$. 

We place a constraint on the example set that $Ex:\{(\tau_i,\tau_o)\}$, or more specifically $\forall (i,o) \in Ex,\ i:(\tau_i \in T) \land o:(\tau_o \in T)$.
In practice this a trivial constraint, achieved by requiring the user to provide the types explicitly \cite{Osera:2015} or to infer the types \cite{gulwani_popl15} based on regular expressions.



\subsection{Lifting Example Types}
While conceptually simple, enumerating all well-typed functions is wasteful - if possible we need to prune the search space.
To do this, we can exploit an unstated assumption, but widely accepted approach, in existing programming-by-example work.
Usually, the example pair type is lifted into a function type in the trivial way.

\begin{flalign*}
lift\ (\tau_i, \tau_o) =\ \tau_i \to \tau_o
\end{flalign*}

However, a subtyping relation can create more specific types that will better prune the space.
The subtyping relation $A<:B$ means that any time type $B$ results in a well-typed program, so would type $A$ in place of $B$.
A subtyping relation induces a subset relation of terms of type $A$ in relation to the terms of type $B$.
Given $\mathcal{A} = \{ x | x::A\}$ and $\mathcal{B} = \{ x | x::B\}$, we have $\mathcal{A}\subseteq\mathcal{B}$.
Lifting the examples to a subtype of the trivial lifting can then yield a smaller search space.

\begin{flalign*}
lift'\ (\tau_i, \tau_o) <:\ \tau_i \to \tau_o\\
\end{flalign*}

Notice that we did not write out a full function for the subtype.
This would have implied a subtyping on the component types, specifically the inputs and outputs would be contravariant or covariant, respectively.
However, we do not wish to restrict the domain or range of the function, but only the size of function space.
So we must have the following

\begin{flalign*}
lift'\ (\tau_i, \tau_o) =&\ \tau^{s}_{i} \to \tau^{s}_{i} \nRightarrow\\
(\tau^{s}_{i} <: \tau_i)\ \lor&\ (\tau_o <: \tau^{s}_{o})
\end{flalign*}


As an demonstration of this approach, following the syntax from previous code samples, we demonstrate below the synthesis of \codeinline{map (+1)}. We provide examples of type \codeinline{([Int],[Int])}.
\begin{lstlisting}
exs :: [ [Int] :-> [Int] ]
exs = [
  [1]   :-> [2],
  [3,4] :-> [4,5] ]
\end{lstlisting}

Using the traditional approach, we would have the trivial lifting to the function type.
However, if we use $lift'$ instead, we would derive a more specific type.
This more specific type could be a refinement type, expressed here using the syntax of LiquidHaskell\cite{DBLP:conf/icfp/VazouSJVJ14}.
 
\begin{lstlisting}
exs :: [Int],[Int]
lift(exs) :: [Int] -> [Int] 
lift'(exs) ::
  {x:[Int] -> y:[Int] | length x > length y}
\end{lstlisting}

Alternatively, we could also have derived the equally specific type using dependant types\cite{dependant_types}.
In this case, we would need a definition of (\codeinline{Vec L a}) to describe the type of lists of length \codeinline{L} and elemental type \codeinline{a}.

\begin{lstlisting}
lift'(exs) ::
  {Vec L Int -> Vec L Int}
\end{lstlisting}


Notice that in either case, the target function type is a subtype of the trivial lifting, but we have not changed either the types of either the domain or range of the target function.

Our approach to natural synthesis is then: given a type environment $\Gamma$ and an example set $Ex:\{(\tau_i,\tau_o)\}$, enumerate $\mathcal{G}$ in order of naturalness, such that $e : lift'(\tau_i,\tau_o)$.
This list can then be checked in order to find an $e$ such that $\forall (i,o) \in Ex, e (i) = o$.

\subsection{function classification}
we need to assign all functions in $\mathcal{G}$ a refinement type in order to use the above result.
Hence we do offline analysis.



%
%This framing will draw the type of the target expression directly from the examples.

\subsection{Solution Space}\label{solnSpace}
While this approach can work for first-order synthesis, we instead focus on data structure manipulation problems that can be solved with higher order functions.

We require all higher order functions to be of a unified signature \texttt{$\_ \to * \to *$}, where the final kind of the signature is a function mapping the input type to the output type. 
Here, a kind is understood to be the type of a type constructor, in this case \texttt{$\to$}, which constructs a function type from two other types.

The practical consequence of this format is that a user must partially uncurry (collapsing trailing function arguments into a single tuple argument) any higher-order function they are interested in using during synthesis.
This also means that any type variable appearing in the higher-order function must be accounted for in the input and output types so that all type variables in its signature can be resolved.
This allows us to conclude that any types that are between the input and first order function will be static initial values, which can be assigned using the process described in Section \ref{makeFxns}.
This is a simple procedure that makes use of the user's domain knowledge of which parameters to the function will be given by the examples; consider:

\begin{lstlisting}
zipWith' :: (a -> b -> c) -> ([a], [b]) -> [c]
zipWith' f (xs,ys) = zipWith f xs ys
\end{lstlisting}

By formally defining the space of functions we are interested in synthesizing, we can this definition to prove some properties on the algorithm.
In particular we show in Section \ref{sound} that \ourTool/ is complete for this subset of functions.

the solutions \ourTool/ supports synthesizing are higher-order data structure manipulation programs.
The higher-order functions take a component function that is a first-order function, for example \codeinline{(+)}.
The solution programs can be expressed as:
% up to reordering of terms (we dont actually support this, should we really include this)

\begin{lstlisting}
solution ::
           (* -> types)  -- Component Function
        ->  types        -- Initial Values
        ->  *            -- Input
        ->  *            -- Output
types = * | * -> types
-- * matches on type variables and constructors.
\end{lstlisting}

Generally, the component function is applied across the \textsf{input} data structure, which the \textsf{solution} uses to construct an \textsf{output} data structure or reduction. As we will argue in Section \ref{evaluation} this set is expressive enough to support the classic \texttt{map}, \texttt{filter}, and \texttt{fold} functions, as well as higher order functions found in imported modules and user-supplied code.

Our goal is to create a synthesis procedure that is easily portable across full implementations of functional languages (Haskell, OCaml, etc), so we prefer using a type directed approach to synthesis over explicit code analysis whenever possible. This increases the portability and longevity of our system. For this implementation we target Haskell, detailing the exact modifications needed to expand this to other languages in Section \ref{languageSupport}.

%Our algorithm does not explicitly try to fit component functions to the examples. Instead, we leverage a promising body of existing work in synthesizing top-level, first-order functions \cite{potential, reviewers}. While it is out of scope to go in to detail, we will briefly discuss the integration of these synthesis procedures in Section \ref{conclusions}.

%The liquidHaskell predicate applied to this signature will be of the effect of \texttt{len([a],[b]) = len([c])}.


\input{overview}

\input{offline}

\section{Online: Fitting Functions to Examples} \label{synth}
\input{online0}

\section{Evaluation}\label{evaluation}
\section{Evaluation}\label{evaluation}

\subsection{Soundness and Completeness}\label{sound}

It is clear that no function will be returned by the algorithm that does not fit the examples given, since functions are validated before being reported.
Therefore, it is trivial to conclude that \ourTool/ is sound over the given examples.
Still, it is possible for the synthesis procedure to return a function that does not capture the user's intent - that is, as with any programming by examples system, \ourTool/ is not sound over the user intent.
Generally, this ambiguity can be resolved by the user supplying more examples to narrow the set of possible fitting functions.
However, depending on what the user is trying to synthesize, and which examples have been provided, it is possible for new examples to increase the internal search space.
If, for example, a user gives only positive examples for a \texttt{filter}, the refinement type predicate discovery will assume that the lists do not change size, and will likely return \texttt{map id} as a result.

Since we perform enumerative search over $\mathcal{G}_I$, which is a subset of the finite set $\Gamma^2$, our approach is complete in the trivial sense. 
If the search space was the full $\mathcal{G}$, we would have completeness (every permutation of every identifier in scope is equivalent to the full language).
This is not at all a useful completeness, since the generated programs would be unreadable.
For example, if the constant 10 is not in $\mathcal{G}$, we can only reproduce it with some combination of operations on the hardcoded initial values, e.g. (1+...(1+ (1))).



%The completeness claim we might like to make is that over the solution space defined in Section \ref{problem}, we will always find a solution if it exists.
%Since our space is finite (TODO considering only initial values induced by the types monoid instance), completeness can be made trivially true by replacing all instances of pruning with a zero ranking, so that our algorithm now is only a best-first enumerative search.
%Because we make some decisions in pruning that removes potentially sound functions, such as using the \codeinline{noRType} tag in Section \ref{HORtypeInf} we trade this completeness for performance.
%In Section \ref{sec:related}, we will discuss why, even if we had completeness, it should be sacrificed in future work.
%On the other hand, the set of functions that the algorithm can produce is fairly broad. It is able to search through the entire space of higher order functions that have been specialized with a first-order function, when considering the functions that are in scope. We will see in Section \ref{evaluation} how broad this space actually is. \markk{See Section \ref{solnSpace}}


\subsection{Performance}

\begin{table*}[t]
  \centering
  \bgroup
  \def\arraystretch{1.1}
  \begin{tabular}{|c|l|c|l|l|l|l|}
    \hline
    & Name & Time (s) & Imports & \# Ex & Representative Example & Generated Function \\
    \hline
    \parbox[t]{2mm}{\multirow{4}{*}{\rotatebox[origin=c]{90}{Bool}}}
    & and & 0.87 & None & 3 & [True, False] $:\to$ False & all id \\
    & and (2nd) & 4.05 & None & 3 & [True, False] $:\to$ False & foldl min True \\
    & or  & 1.84 & None & 4 & [True, False] $:\to$ True & any id \\
    & xor & 3.13 & None & 4 & [True, False, True] $:\to$ False & foldl xor False \\
    \hline

    \parbox[t]{2mm}{\multirow{4}{*}{\rotatebox[origin=c]{90}{Tree (u.d.)}}}
    & double vals & 4.02 & None & 1 & ((1) 3 (2)) $:\to$ ((2) 6 (4)) & mapBTree (*2) \\
    & tree id & 3.00 & None & 1 & ((1) 3 ((4) 5 (6))) $:\to$ ((1) 3 ((4) 5 (6))) & mapBTree id \\
    & tree max & 3.84 & None & 3 & ((1 10) 5) $:\to$ 10 & accumTree max 1 \\
    & tree sum & 0.72 & None & 1 & ((3 1) 2) $:\to$ 6 & accumTree (+) 0 \\
    \hline

    \parbox[t]{2mm}{\multirow{9}{*}{\rotatebox[origin=c]{90}{List}}}
    & all even & 1.27 & Data.List & 4 & [2,4,6,8] $:\to$ True & all even \\
    & some odd & 2.91 & Data.List & 3 & [1,4,5,6] $:\to$ True & any odd \\
    & custom filter & 15.63 & Data.List & 3 & [1,2,3,4,5] $:\to$ [3,4,5] & filter user\_pred \\
    & length & 1.05 & Data.List & 3 & [5,6,7,8] $:\to$ 4 & foldl count 0 \\
    & max elem & 3.54 & Data.List & 3 & [4,10,7] $:\to$ 10 & foldl max 0 \\
    & negate all & 12.58 & Data.List & 1 & [True, False, True] $:\to$ [False, True, False] & map not \\
    & odd prefix & 30.62 & Data.List & 1 & [1,3,4,6,7] $:\to$ [1,3] & takeWhile odd \\
    & stutter & 5.72 & Data.List & 1 & [1,2,3] $:\to$ [1,1,2,2,3,3] & concatMap (replicate 2) \\
    & sum ints & 0.84 & Data.List & 1 & [1,2,3,4] $:\to$ 10 & foldl (+) 0 \\
    \hline

    \parbox[t]{2mm}{\multirow{4}{*}{\rotatebox[origin=c]{90}{Etc.}}}
    & set sum & 0.90 & Data.Map & 1 & \{ 1, 2, 3, 4 \} $:\to$ 10 & Data.Map.foldl (+) 0 \\
    & music id & 3.48 & Euterpea & 1 & C\# $:\to$ C\# & mMap (fromIntegral) \\
    & transpose score & 4.69 & Euterpea & 1 & A $:\to$ B & mMap (trans 2) \\
    \hline
  \end{tabular}
  \egroup
  \caption{Benchmarks and Performance Measures. This table lists all 20 benchmarks, grouped by data structure. Each benchmark lists its name, the amount of time it took to synthesize, the extra imports it uses, the number of examples needed to synthesize, one representative example, and the synthesized function itself. The group marked ``Tree (u.d.)'' is a user-defined structure with user-defined higher-order operations. All reported data is generated on a Linux machine with four cores of Intel Xeon E5-2650Lv3 @ 1.80GHz and 8 Gb of ram.}
  \label{tab:benchmarks}
\end{table*}

In Table \ref{tab:benchmarks} we show detailed information about \ourTool/ over a set of benchmarks. These benchmarks were chosen to show the versatility of \ourTool/ over many different applications and libraries. The benchmarks over Booleans, trees, and lists are common to many other programming-by-example tools. The examples that utilize the \codeinline{Data.List} and \codeinline{Euterpea} libraries to show \ourTool/'s ability to work with large, highly specialized, 3rd-party libraries. Due to the algorithm's focus on generating natural code, the synthesized functions are concise enough to be listed within the table itself. The representative examples show that few, simple hints to the synthesizer are able to produce good results. In many cases, the representative examples are actually the \textit{only} examples necessary to synthesize the desired function. This shows that our approach uses the information available to it effectively.

In addition, the runtime average about ten seconds thanks to the inherently parallel nature of the search. With just a few lines of code, we were able to achieve order-of-magnitude speedups over the serial version. Haskell's functional parallelism model is ideal for embarrassingly parallel problems like this one, and promises good scaling to larger instances of the problem over increasing computational resources.

In Section \ref{HORtypeInf} we discuss how type matching and the \codeinline{noRType} tag reduce the number of refinement type inferences we make. Recall that even if both types have a measure (lists and trees), in general there is no guarantee that this is a meaningful comparison. Since \lhask/ is the largest cost to our system in the offline stage, removing refinement type inference in these ambiguous cases provides a large performance gain. As an example, in processing the Haskell standard library \codeinline{base:Prelude}, 7 out of 30 higher order functions do not need to be checked against refinement types using this approach.


\subsection{Example Generation}\label{languageSupport}


Our goal is to create a synthesis procedure that is easily portable across full implementations of functional languages (Haskell, OCaml, etc), so we prefer using a type directed approach to synthesis over explicit code analysis whenever possible.
This increases the portability and longevity of our system.
To this end, we have tried to avoid code analysis at every stage of this paper.
However there are two points where this has fallen short.

First, we must parse a file to extract the name and type information of every top level identifier.
Second, using \lhask/ as a blackbox means that we rely on \lhask/'s mechanisms to check refinement types over functions.
Our eventual goal is to create a system that can be easily ported across functional languages.
Luckily, the dependency on extracting type information is small enough to handle with ease in most typed languages (the grammar of a type signature is relatively small).
However \lhask/ is a powerful tool that would be difficult to recreate in another language.

One approach to solve this is to extend the refinement type system by allowing refinement type inference on representative examples of a higher order function.
We do not need to identify a particular component function since we are only interested in size based refinement types.
We then apply a similar refinement type inference strategy as in Listing \ref{exRTypeGen} to these examples.

Our current example generation tool uses QuickCheck to generate and apply many examples for higher order functions in Haskell~\cite{quickcheck}.
Since \lhask/ supports so much of Haskell, this was not a necessary extension for \ourTool/, but provides a prototype as a proof of concept.
Imagining that we could not find a refinement type directly on \codeinline{filter}, we might instead generate examples on and use them to infer a refinement type - just as we infer a refinement type on the user-provided examples.

However, we are not guaranteed to generate a correct refinement type because we might not generate fully representative examples.
If the tool may not generate examples where the predicate on the filter is used, creating a situation equivalent to \codeinline{map id}.
In this case we would incorrectly infer the refinement type \codeinline{filter :: _ -> i:[a] -> \{o:[b] | (len i) = (len o)\}}

\section{Related Work}
\label{sec:related}

In describing \ourTool/, we have introduced the problem of natural synthesis. 
We have shown how to formulate an enumerative type-directed approach synthesis that results in natural code, but also frees us from burdensome constraints of code analysis and hard coded inference rules.
Many of the techniques we have used have been explored in various contexts before, though generally for the purpose of lower level synthesis. 
In this section we make some comparisons to related work, and highlight the differences we employ that help us generate readable code.

One of the most closely related works in aspirations is a tool called MagicHaskeller~\cite{DBLP:conf/aaip/Katayama09}. This project makes heavy use of a ranking system based on code use and lookup frequency in a database to deliver natural results to the user. In contrast to our work, MagicHaskeller uses a database of functions as its main synthesis engine, with the current database hovering around
64GB~\cite{DBLP:conf/agi/Katayama15}. From this work, we take the inspiration of supporting imported libraries for creating natural code. However, it is important that the system is more portable and easily manipulated by the user - in particular by allowing user defined function in synthesis.

MagicHaskeller works in the same AI focused domain of inductive programming as the tool IgorII~\cite{DBLP:conf/aaip/HofmannKS09}. IgorII however takes a very code analysis heavy approach, having been originally developed for Maude, then ported to Haskell. Although these approaches tend to generate readable code, there has not yet been formal analysis of this feature.

One of the motivating works for exploring type-directed programming by example, especially over recursively defined datatypes, is MYTH~\cite{Osera:2015, Osera:2016}. The natural extension of this work in the usability direction was to include a more lightweight and flexible support for user defined and imported datatypes. The $\Lambda^2$ tool also focuses on deriving programs over recursively defined datatypes~\cite{Feser:2015}. One of the major barriers to an average user with these tools, is that the generated code operates on the inner workings on a datatype. While this provides a complete picture of all the data manipulation, often a user might prefer to simply be provided with high level, functioning code. Building in support and the ability to reason on user defined functions in \ourTool/ has made this natural synthesis possible.

\cite{Osera:2016} ``does not infer instantiation of polymorphic library functions, and the theoretical framework restricts the combination of union and intersection types''.
This limitation is due to the ``programs as proofs'' approach to synthesis.
That approach is more complex than enumeration, and unsurprisingly faster, but is limited by a heavy theoretical framework.
However, the approach in ~\cite{Osera:2016} can solve many synthesis problem because of the inclusion of lambda terms in the search space.

Another valuable contribution from~\cite{Osera:2016} is the relationship between refinement types and examples.
This allows the system to use refinement types to extend the specification language of a programming by example system.
Instead, \ourTool/ keeps refinement types entirely as a backend logical inference technique and hidden from the user.
In line with our goal of synthesizing natural code, we wish to minimize  asking the onerous task of users to learn to write new specifications.
Simple as they may be, refinement types are not yet seem as approachable as the more familiar ``examples as a specification'' to the average user.

Taking the refinement types as a specification even further,~\cite{dblp1683325} proposes a system Synquid, that will synthesize programs based on refinement types. 
At first glance this seems to be an entirely different approach than \ourTool/, which instead uses examples as the specification and only uses refinement types as a search space pruning tool.
However, the work in~\cite{Osera:2016} does give an indication that our use of refinement types may be related on fundamental, proof theoretic level. 
By exploring this relationship in more depth, it may be possible to draw a stronger parallel between these various works and port ideas from one system to another.

One of the most widely used and well known instances of a programming by example system being used by many novice users is FlashFill~\cite{GulwaniHS12}.
Like other system, the goal of this work is to make executable, not code. 
This leaves users without the ability to modify generated source code.
In the case of FlashFill, being embedding within production level software, most users are not clamoring for this feature.
However it would certainly open an interesting avenue to introduce new users to computer programming if this were an option.
StriSynth~\cite{icse} takes a step in this direction by providing a natural language description of the synthesized program, but the source code remains obfuscated and inaccessible.

One difficult limitation is that without subexample generation we cannot recursively apply our algorithm as in the $\Lambda^2$ tool~\cite{Feser:2015}.
Subexample generation gives the ability to recursively call the synthesis engine to generate programs with multiple applications of high order functions.
However, since the ability of $\Lambda^2$ to generate subexamples relied on hard coded subexample generation hypotheses for the predefined set of higher order functions, this does not scale.
While inferring the hypotheses might be possible by inspecting the code, we have maintained a dedication to minimize our reliance on code analysis techniques for portability and longevity of the system.
The best way to automatically create subexample generation functions solely based on type information remains a difficult problem.








\section{Conclusions}
\label{conclusions}
Even for novice computer users, the need for basic programming skills is increasing. To meet this need, synthesis tools must be able to produce code that is not only correct, but also useful to the users of these tools. Program synthesis is powerful, but much of this power is left untapped when results are obfuscated by level upon level of folds, filters, and maps. Correctness is only a baseline, if program synthesizers do not produce natural code, most users will not be able to take full advantage of this field.

There are many directions left to explore. Enhancing our refinement type inference from examples will allow greater search space pruning for practical efficiency benefits. The offline / online structure of our algorithm allows novice users to interact with the useful synthesis part of our algorithm without dealing with the complicated refinement-type formulation of our search space pruning, so our tool would integrate well into a live programming environment. Finally, the use of a more powerful language like Agda should be considered as an easier way to manipulate the type directed search.

Finally, as we saw in Section \ref{evaluation}, our tool responds to synthesis queries with highly-readable, transparent, and ultimately natural functions that fit the examples provided. We believe our tool is a step toward realizing a robust and accessible program synthesis system.


\bibliographystyle{abbrvnat}
\bibliography{myBib}


\end{document}

\section{Problem Formulation}
\label{problem}

\subsection{Naturalness}

The goal in natural synthesis is synthesize programs that meet a specification and maximizes naturalness.
For programming by example, the specification is a set ($Ex$) of pairs of input and output $(i,o)$.

We define \textit{naturalness} of a code section C to simply be the inverse of number of expressions in C.
Expressions are variables, function applications, operators.
Taking the code from Listing \ref{natSyn}, solution1 has a naturalness score of 1/8, while solution2 has a score of 1/3.
The intention in this definition is to reduce the volume of generate code.
This favors the use of predefined functions over the generation of lambda terms and low level function composition.
This is a very subjective measure and impossible to argue without extensive user studies.
We will simply remark that this seems to be roughly accurate, especially for novel programmers, a primary target for \ourTool/.

The problem is then to find an expression e, such that $\forall (i,o) \in Ex, e (i) = o \land min(\mathcal{N}(e))$.


\subsection{Enumeration}
In order to solve the above problem, we will immediately restrict our search space to not include generated lambda terms, as such terms will generally induce a very low naturalness score and explode the search space.
Our synthesis approach will then only be able to solve synthesis problems when a solution exists that only uses a finite set of predefined values.

Let $E$ be the finite set of expressions exposed to the top-level module from user code, imported libraries, and the core library.
Our search space will be the set of permutations of well-typed applications of elements of $E$.

To determine if an expression is well-typed, let $T$ be the set of types (both inferred and explicitly declared) exposed to the top-level module from user code, imported libraries, and the core library.
A type environment $\Gamma$, is the set $\{e1 : \tau_1,\ ...,\ en : \tau_n\}$, where $e_{i} \in E$ and is of type $\tau_i \in T$.
The set of well-typed expressions we consider, $\mathcal{G}$, is the set $\{e1 : \tau_1,\ ...,\ en : \tau_n\}$, where $e_i : \tau_i$ follows the usual rules of application for constructing well-typed expressions from $\Gamma$. 

We place a constraint on the example set that $Ex:\{(\tau_i,\tau_o)\}$, or more specifically $\forall (i,o) \in Ex, i:\tau_i \in T \land o:\tau_o \in T$.
In practice this a trivial constraint, achieved by requiring the user to provide the types explicitly \cite{Osera:2015} or to infer the types \cite{gulwani_popl15} based on regular expressions.



\subsection{Lifting Example Types}
An unstated assumption, but widely accepted approach, is to lift the example pair type into a function type in the trivial way.

\begin{flalign*}
lift\ (\tau_i, \tau_o) =&\ \tau_i \to \tau_o
\end{flalign*}

However, we can do much better than that by using refinement types to restrict the space.

\begin{flalign*}
lift\ (\tau_i, \tau_o) <:&\ \tau_i \to \tau_o\\
lift\ (\tau_i, \tau_o) =&\ \{\tau_i \to \tau_o\ |\ p\}
\end{flalign*}

Our approach to natural synthesis is then: given a type environment $\Gamma$, an example set $Ex:\{(\tau_i,\tau_o)\}$, enumerate $\mathcal{G}$, in order of naturalness, such that $e : lift(\tau_i,\tau_o)$.
This list can then be checked to find  $e |\ \forall (i,o) \in Ex, e (i) = o$.





\subsection{Crap beyond here}
In enumerative, type directed program synthesis the goal is to find an expression e such that its type matches some $\tau$.
In programming by example, that $\tau$ is a pair type, derived from the examples (which we express with $\leadsto$).
Synthesis engines may either require the user to provide these type explicitly\cite{Osera:2015} or to infer these types \cite{gulwani_popl15}.


\begin{flalign*}
(i,o) \leadsto& (\tau_i, \tau_o)\\
lift\ (\tau_i, \tau_o) =&\ \tau_i \to \tau_o\\
\Gamma, i:\tau_i, o:\tau_o \vdash&\ e \colon liftFun\ (\tau_i, \tau_o)
\end{flalign*}

\subsection{The implicit assumption}\label{sec:inhab}

An \textit{example} is a pair of values with distinguished ``input'' and ``output'' elements, and an \textit{example set} is a set of examples all of whose inputs are of like type, and all of whose outputs are of like type.
The output type does not necessarily match the input type.
Synthesis engines may either require the user to provide these type explicitly\cite{Osera:2015} or to infer these types \cite{gulwani_popl15}.
In either case, the example types are embedded into the arrow type to create the type of the function that must be found. 

\begin{flalign*}
(i,o) \leadsto& (\tau_i, \tau_o)\\
lift\ (\tau_i, \tau_o) =&\ \tau_i \to \tau_o\\
i:\tau_i, o:\tau_o \vdash& e \colon liftFun\ (\tau_i, \tau_o)
\end{flalign*}

We could create an type synonym for the pair with the operator $\mapsto$, to denote an example, and synthesis proceeds in the same way.

\begin{flalign*}
lift\ (\tau_i, \tau_o) <:&\ \tau_i \to \tau_o\\
%i \mapsto o \leadsto& \tau_e\\
%liftFun\ \tau_e <:&\ \tau_i \to \tau_o\\
%i \mapsto o : \tau_e \vdash& e \colon f(\tau_e)
\end{flalign*}

Now, how do we actually create a type that is subtype of arrow.

\begin{flalign*}
\{\tau_i \to \tau_o\ |\ p\} <:&\ \tau_i \to \tau_o\\
%i \mapsto o \leadsto& \tau_e\\
%liftFun\ \tau_e <:&\ \tau_i \to \tau_o\\
%i \mapsto o : \tau_e \vdash& e \colon f(\tau_e)
\end{flalign*}

It is now clear we have lost type information, specifically the relation between the input and output.
Our inference for the target function type should instead use all the information available.


The information encoded in the pair relate
Any type system that allows for type-value level fluidity can be used to implement the inference of Equation 2.
Both refinement types and dependent types could be used. 
In our implementation, we use refinement types to create such a type for its ease of integration with existing Haskell code.

\begin{table}
    \begin{tabular}{l|l}
        \hline
        Type of Code & Weight \\ \hline
        User Defined Value & 100 \\ 
        Library Import &  50\\
    \end{tabular}
\end{table}

The synthesis problem is then to find an expression $e$ of the inferred type, with the additional constraint that $e$ satisfies the examples.
The inferred type using a simple pair is shown in Equation 1, while using the specialized example type inference is shown in Equation 2.
While these seem quite similar, Equation 2 allows for more specific inference that relates the values of input and output at the type level in a more meaningful way than the simple arrow type.
Framing the synthesis problem as a type inhabitation problem where we enumeratively search for an expression of the correct type\cite{DBLP:conf/pldi/GveroKKP13}, the more specific type in Equation 2 can significantly reduce the search space.



%
%This framing will draw the type of the target expression directly from the examples.

\subsection{Solution Space}\label{solnSpace}
Synthesizing correct programs is a well researched problem; however, if programming-by-example is to become a mainstream tool for programmers, the synthesized code must be easy for a human to read and modify.
The aim of \ourTool/ is to synthesize programs from examples that utilize user defined code in a clear and concise.
We focus particularly on data structure manipulation problems that can be solved with higher order functions.


A user supplies examples via a custom pair constructor \texttt{:->}. This operator is used to differentiate between generic pairs and examples, but does not confer any additional structure. We require all higher order functions to be of a unified signature \texttt{$\_ \to * \to *$}, where the final kind of the signature is a function mapping the input type to the output type. Here, a kind is understood to be the type of a type constructor, in this case \texttt{$\to$}, which constructs a function type from two other types.

The practical consequence of this format is that a user must partially uncurry (collapsing trailing function arguments into a single tuple argument) any higher-order function they are interested in using during synthesis.
This also means that any type variable appearing in the higher-order function must be accounted for in the input and output types so that all type variables in its signature can be resolved.
This allows us to conclude that any types that are between the input and first order function will be static initial values, which can be assigned using the process described in Section \ref{makeFxns}.
This is a simple procedure that makes use of the user's domain knowledge of which parameters to the function will be given by the examples; consider:

\begin{lstlisting}
zipWith' :: (a -> b -> c) -> ([a], [b]) -> [c]
zipWith' f (xs,ys) = zipWith f xs ys
\end{lstlisting}

By formally defining the space of functions we are interested in synthesizing, we can this definition to prove some properties on the algorithm.
In particular we show in Section \ref{sound} that \ourTool/ is complete for this subset of functions.

the solutions \ourTool/ supports synthesizing are higher-order data structure manipulation programs.
The higher-order functions take a component function that is a first-order function, for example \codeinline{(+)}.
The solution programs can be expressed as:
% up to reordering of terms (we dont actually support this, should we really include this)

\begin{lstlisting}
solution ::
           (* -> types)  -- Component Function
        ->  types        -- Initial Values
        ->  *            -- Input
        ->  *            -- Output
types = * | * -> types
-- * matches on type variables and constructors.
\end{lstlisting}

Generally, the component function is applied across the \textsf{input} data structure, which the \textsf{solution} uses to construct an \textsf{output} data structure or reduction. As we will argue in Section \ref{evaluation} this set is expressive enough to support the classic \texttt{map}, \texttt{filter}, and \texttt{fold} functions, as well as higher order functions found in imported modules and user-supplied code.

Our goal is to create a synthesis procedure that is easily portable across full implementations of functional languages (Haskell, OCaml, etc), so we prefer using a type directed approach to synthesis over explicit code analysis whenever possible. This increases the portability and longevity of our system. For this implementation we target Haskell, detailing the exact modifications needed to expand this to other languages in Section \ref{languageSupport}.

%Our algorithm does not explicitly try to fit component functions to the examples. Instead, we leverage a promising body of existing work in synthesizing top-level, first-order functions \cite{potential, reviewers}. While it is out of scope to go in to detail, we will briefly discuss the integration of these synthesis procedures in Section \ref{conclusions}.

%The liquidHaskell predicate applied to this signature will be of the effect of \texttt{len([a],[b]) = len([c])}.

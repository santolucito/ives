\section{Online: Fitting Functions to Examples} \label{synth}

With the synthesis engine constructed (all of $\Gamma$ has been assigned an inferred refinement type), the system is ready to synthesize programs from examples.
That is, we are ready to construct the set of expressions $\mathcal{G}_I$.
Multiple programming-by-example queries can then be answered using this engine.
The synthesis engine only needs to be reconstructed when there are new library imports, or when there is a revision of the user-supplied code.

When examples are provided, the synthesis engine finds a suitable refinement type for a hypothetical function that could fit that example.
Then, \ourTool/ filters and ranks the higher order functions based on the refinement types known to the engine and the example types provided.
Once the candidate higher functions are identified, \ourTool/ will select and build first order functions that match the type of the higher order function's component signature to build a final set of candidate programs.

Each of these candidate programs is executed in best-first order against the set of inputs.
The ordering here based on the choices we made during construction of this list - in particular the type match ranking.
This is both an approximation ordering of naturalness and an ordering on likelihood of fitting the examples.
Whenever a function produces the correct outputs for each input, it is said to fit, and is reported to the user.
This search continues until the space is exhausted or it is manually interrupted.
The search will always terminate since we are working over a finite space of generated functions, are our type reductions are strictly decreasing, which we will explain in Section \ref{typeMatch}

% getExampleType
% assignRType
\subsection{Refinement types for examples}
% getExampleType
As in Section \ref{HORtypeInf}, we also consider two cases for examples. The first, where the example input and output types match up to the top level type constructor, and the the case where the types do not match.

% assignRType
In the case that the types do match, we find the set of refinement types that the examples satisfy. Generating refinement type predicates about the size of the input and output, as in Section \ref{HORtypeInf}, we apply the same algorithm from Listing \ref{listing:addRType}. 
For instance, an example set for \codeinline{filter (>3)} might look as follows:

\begin{lstlisting}[caption=Refinement type inference for examples,label=exRTypeGen]
exs :: ([Int] , [Int])
exs = [([1,2,3] , [1,2,3]),
      ([1,3,4] , [1,3]),
      ([4,6,8] , [])]
       
exsRType ::
  (i : [Int], { o : [Int] |
  len i >= len o })
\end{lstlisting}

\noindent and have the final refinement type of \codeinline{exRType}, since all of the examples suggest that the output list does not grow. 
Again, when the types do not match we assign the \codeinline{noRType} flag to the examples, as we did for higher order functions in Listing \ref{listing:addRType}.
We can now reduce our search space to only higher order functions with the same refinement type that matches the examples' refinement type. 


\subsection{Type match ranking}\label{typeMatch}

Once \ourTool/ has both the base and refinement types for the examples and higher order functions, it can can prune and order this set (line 17 of Listing \ref{listing:Algo}).
The first step is to simply filter the higher order function candidates over equality of refinement types.
Additionally, \ourTool/ will check the example types are concrete versions of the input/output types of the higher order function with the infix (for clarity) \codeinline{isConcreteTypeOf} function.
For type A to be a concrete version of type B, there must exist some type C (possibly equal to type B), such that both A and B can be instantiated to that type.
The above requirement is then that there is some way to unify these two types - a familiar problem\cite{typeUnif}.

\begin{lstlisting}[caption=Pruning based on types]
filter (exRType ==) higherOrderRTypes
filter (exType `isConcreteTypeOf') higherOrderComponentTypes
\end{lstlisting}

Once these higher order functions have been culled from the pool of candidates, we update their ranks that had been assigned in Section \ref{HORtypeInf} from code locality.
The higher order function can advance in the ranking by using a value function to find out exactly how much the example type \codeinline{isConcreteTypeOf} to the input/output types of candidate higher order function.

In Listing \ref{valueAlgo}, we present a demonstration of part of this ranking algorithm.
As we traverse the tree structure of the type, the more pieces of the type signature that match, the higher the value of that match. 
However, if there is a type constructor mismatch, the two types can never be reconciled, and the entire value gets nothing.

\begin{lstlisting}[caption=Type closeness ranking algorithm (sample),label=valueAlgo]
value :: Type -> Type -> Maybe Int
value (TyFun i1 o1) (TyFun i2 o2) = fmap (1+) 
    (liftA2 (+) (value i1 i2) (value o1 o2))
value (TyCon n1) (TyCon n2) =
   if (n1==n2) then Just 20 else Nothing
value (TyCon n1) (TyVar _) = Just 10
value _ _ = Nothing
\end{lstlisting}

As an example of how this value function is applied the higher order functions, imagine we have three map functions specialized on particular values. 
The fully polymorphic map will score 1 point for having a function between input and out, 2 points for both having lists, and 20 points for a type variables matching a type constructor, for a total of 5 points. The mapI for Ints, will score the same, but score 20 points for each matching type constructors instead of 10 points for each type variable matched to a type constructor. The mapB for Boolean value gets nothing since there is no way to reconcile that type to the example type.

\begin{lstlisting}[caption=Ranking higher order function,label=horank]
exs ::                   ([Int] , [Int])
map  :: (a    -> b)    -> [a]    -> [b]
mapI :: (Int  -> Int)  -> [Int]  -> [Int]
mapB :: (Bool -> Bool) -> [Bool] -> [Bool]

-- map  scores 5
-- mapI scores 43
-- mapB scores Nothing
\end{lstlisting}


\subsection{Component function generation}\label{makeFxns}
% makeFxns

\ourTool/ must now find first order function for each of the higher order functions that are still candidates (line 18 of Listing \ref{listing:Algo}), in order to create complete functions in $\mathcal{G}_I$.
For a given higher order function, \ourTool/ can choose component functions by reusing the weighted type matching algorithm from Listing \ref{valueAlgo}.
Since examples must be given as a concrete type, we can always partially specialize our candidate higher order function. 
We then search for first order functions that will type check against the partially specialized component signature.
This partial specialization is a way of extracting more information out of our examples, and significantly reduces the space of candidate first order functions.
Similar to Listing \ref{horank}, we show an example of how type matching is applied over first order functions in Listing \ref{comprank}.

\begin{lstlisting}[caption=Ranking component function,label=comprank]
exs   ::                ([Int] ,  [Int])
map   :: (a   -> b)   -> [a]   -> [b]
mapEx :: (Int -> Int) -> [Int] -> [Int]

component ::
      Int    -> Int
f1 :: a      -> b      -- value is 21
f2 :: Int    -> a      -- value is 31
f3 :: Int    -> Int    -- value is 41
f4 :: [Bool] -> [Bool] -- value is Nothing
\end{lstlisting}

\subsection{Initial Values}\label{initVals}

In addition to considering first order functions where the arity of the type signature is equal to the component function, we may also want ``larger'' functions that have been applied to initial values.
For examples, if the component signature is \codeinline{::Int->Int}, we may have the first order functions \codeinline{(+)::Int->Int->Int} in scope.
By applying an initial value to \codeinline{(+)}, we can get a new function (e.g. \codeinline{(+1)::Int->Int}) that fits the component signature.

In order to use an initial value, it must be in scope, i.e. in the set $\Gamma$.
Initial values can be placed into this set in a few ways.
First, users may have some domain knowledge that a particular value, or set of values, may be useful in their application.
In this case, the value only needs to be defined in the same file as the examples.
Users may also write their own specializations of the higher order functions if the value should only be used in the context of a single function.

This approach also handles the case when higher order functions to need initial values in addition to a component function.
For example, the \codeinline{map} function only takes a first order function, while \codeinline{foldl :: (a-> b-> a)-> a-> [b]-> a} requires an initial value for \codeinline{a}.
Using a similar process as for first order function application, we can apply values until the higher order function only needs the example input to complete execution.
To identify initial values in a higher order type signature, we can use our previous assumption that all higher order function have been partially curried to the type \codeinline{_ -> *-> *}. 
Adding the further assumption that only one first order function maybe be passed to the higher order function, we simply tag any non-function type in the hole as an initial value.

\begin{lstlisting}[caption=adding default initial values]
-- to use 5 as an initial value for foldl
fold5 :: (a -> b -> a) -> [b] -> a
fold5 f i o = foldl f 5 i o

-- to use 5 as an initial value for all functions
initVal :: Int
initVal = 5
\end{lstlisting}

Another interesting way to obtain initial values is through the Monoid class.
If the initial value's type is an instance of Monoid, then the identifier mempty, from Haskell's monoid typeclass\cite{monoid}, will be included in $\Gamma$.
As an example, importing the Data.Monoid library, will bring into scope the unit element is for lists, \codeinline{mempty= []}.
There is no Monoid instance for Int in Data.Monoid however. 
There are two valid monoids for numbers, using either (+) or (*) as the operators and resulting in unit elements 0 and 1 respectively. 
Because these are particularly common value,  we hard code both of these values (along with the other useful values of -1, and 2), bringing them into $\Gamma$ and thereby making them available by default in synthesis.

Presented with the problem of finding integer values to satisfy the examples may initially seem like a good application for an SMT solver.
However, keep in mind that we do not in general know what we are trying to solve - the actual use of these variables is hidden within the function definition.
Since in this work we maintain a primarily type directed approach, rather than code analysis, we will not be able to unravel these functions.



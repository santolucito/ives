\subsection{Type match ranking}\label{typeMatch}

Once \ourTool/ has both the base and refinement types for the examples and higher order functions, it can can prune and order this set (line 17 of Listing \ref{listing:Algo}).
The first step is to simply filter the higher order function candidates over equality of refinement types.
Additionally, \ourTool/ will check the example types are concrete versions of the input/output types of the higher order function with the infix (for clarity) \codeinline{isConcreteTypeOf} function.
For type A to be a concrete version of type B, there must exist some type C (possibly equal to type B), such that both A and B can be instantiated to that type.
The above requirement is then that there is some way to unify these two types - a familiar problem\cite{typeUnif}.

\begin{lstlisting}[caption=Pruning based on types]
filter (exRType ==) higherOrderRTypes
filter (exType `isConcreteTypeOf') higherOrderComponentTypes
\end{lstlisting}

Once these higher order functions have been culled from the pool of candidates, we update their ranks that had been assigned in Section \ref{HORtypeInf} from code locality.
The higher order function can advance in the ranking by using a value function to find out exactly how much the example type \codeinline{isConcreteTypeOf} to the input/output types of candidate higher order function.

In Listing \ref{valueAlgo}, we present a demonstration of part of this ranking algorithm.
As we traverse the tree structure of the type, the more pieces of the type signature that match, the higher the value of that match. 
However, if there is a type constructor mismatch, the two types can never be reconciled, and the entire value gets nothing.

\begin{lstlisting}[caption=Type closeness ranking algorithm (sample),label=valueAlgo]
value :: Type -> Type -> Maybe Int
value (TyFun i1 o1) (TyFun i2 o2) = fmap (1+) 
    (liftA2 (+) (value i1 i2) (value o1 o2))
value (TyCon n1) (TyCon n2) =
   if (n1==n2) then Just 20 else Nothing
value (TyCon n1) (TyVar _) = Just 10
value _ _ = Nothing
\end{lstlisting}

As an example of how this value function is applied the higher order functions, imagine we have three map functions specialized on particular values. 
The fully polymorphic map will score 1 point for having a function between input and out, 2 points for both having lists, and 20 points for a type variables matching a type constructor, for a total of 5 points. The mapI for Ints, will score the same, but score 20 points for each matching type constructors instead of 10 points for each type variable matched to a type constructor. The mapB for boolean value gets nothing since there is no way to reconcile that type to the example type.

\begin{lstlisting}[caption=Ranking higher order function,label=horank]
exs ::                   ([Int] , [Int])
map  :: (a    -> b)    -> [a]    -> [b]
mapI :: (Int  -> Int)  -> [Int]  -> [Int]
mapB :: (Bool -> Bool) -> [Bool] -> [Bool]

-- map  scores 5
-- mapI scores 43
-- mapB scores Nothing
\end{lstlisting}

\subsection{Component function generation}\label{makeFxns}
% makeFxns

\ourTool/ must now find first order function for each of the higher order functions that are still candidates (line 18 of Listing \ref{listing:Algo}), in order to create complete functions in $\mathcal{G}_I$.
For a given higher order function, \ourTool/ can choose component functions by reusing the weighted type matching algorithm from Listing \ref{valueAlgo}.
Since examples must be given as a concrete type, we can always partially specialize our candidate higher order function. 
We then search for first order functions that will type check against the partially specialized component signature.
This partial specialization is a way of extracting more information out of our examples, and significantly reduces the space of candidate first order functions.
Similar to Listing \ref{horank}, we show an example of how type matching is applied over first order functions in Listing \ref{comprank}.

\begin{lstlisting}[caption=Ranking component function,label=comprank]
exs   ::                ([Int] ,  [Int])
map   :: (a   -> b)   -> [a]   -> [b]
mapEx :: (Int -> Int) -> [Int] -> [Int]

component ::
      Int    -> Int
f1 :: a      -> b      -- value is 21
f2 :: Int    -> a      -- value is 31
f3 :: Int    -> Int    -- value is 41
f4 :: [Bool] -> [Bool] -- value is Nothing
\end{lstlisting}

\subsection{Initial Values}\label{initVals}

In addition to considering first order functions where the arity of the type signature is equal to the component function, we may also want ``larger'' functions that have been applied to initial values.
For examples, if the component signature is \codeinline{::Int->Int}, we may have the first order functions \codeinline{(+)::Int->Int->Int} in scope.
By applying an initial value to \codeinline{(+)}, we can get a new function (e.g. \codeinline{(+1)::Int->Int}) that fits the component signature.

In order to use an initial value, it must be in scope, i.e. in the set $\Gamma$.
Initial values can be placed into this set in a few ways.
First, users may have some domain knowledge that a particular value, or set of values, may be useful in their application.
In this case, the value only needs to be defined in the same file as the examples.
Users may also write their own specializations of the higher order functions if the value should only be used in the context of a single function.

This approach also handles the case when higher order functions to need initial values in addition to a component function.
For example, the \codeinline{map} function only takes a first order function, while \codeinline{foldl :: (a-> b-> a)-> a-> [b]-> a} requires an initial value for \codeinline{a}.
Using a similar process as for first order function application, we can apply values until the higher order function only needs the example input to complete execution.
To identify initial values in a higher order type signature, we can use our previous assumption that all higher order function have been partially curried to the type $(\star \to \tau_{in} \to \tau_{out})$. 
Adding the further assumption that only one first order function maybe be passed to the higher order function, we simply tag any non-function type in the hole as an initial value.

\begin{lstlisting}[caption=adding default initial values]
-- to use 5 as an initial value for foldl
fold5 :: (a -> b -> a) -> [b] -> a
fold5 f i o = foldl f 5 i o

-- to use 5 as an initial value for all functions
initVal :: Int
initVal = 5
\end{lstlisting}

Another interesting way to obtain initial values is through the Monoid class.
If the initial value's type is an instance of Monoid, then the identifier mempty, from Haskell's monoid typeclass\cite{monoid}, will be included in $\Gamma$.
As an example, importing the Data.Monoid library, will bring into scope the unit element is for lists, \codeinline{mempty= []}.
There is no Monoid instance for Int in Data.Monoid however. 
There are two valid monoids for numbers, using either (+) or (*) as the operators and resulting in unit elements 0 and 1 respectively. 
Because these are particularly common value,  we hard code both of these values (along with the other useful values of -1, and 2), bringing them into $\Gamma$ and thereby making them available by default in synthesis.
Note that using initial values in fact is a slight transgression of the formalism, as this may result in some use of the $\Gamma^3$ space.

Presented with the problem of finding integer values to satisfy the examples may initially seem like a good application for an SMT solver.
However, keep in mind that we do not in general know what we are trying to solve - the actual use of these variables is hidden within the function definition.
Since in this work we maintain a primarily type directed approach, rather than code analysis, we will not be able to unravel these functions.


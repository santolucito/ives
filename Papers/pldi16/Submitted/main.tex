%
% LaTeX template for prepartion of submissions to PLDI'16
%
% Requires temporary version of sigplanconf style file provided on
% PLDI'16 web site.
% 
\documentclass[pldi]{sigplanconf-pldi16}
% \documentclass[pldi-cameraready]{sigplanconf-pldi16}

%
% the following standard packages may be helpful, but are not required
%
\usepackage{SIunits}            % typset units correctly
\usepackage{courier}            % standard fixed width font
\usepackage[scaled]{helvet} % see www.ctan.org/get/macros/latex/required/psnfss/psnfss2e.pdf
\usepackage{url}                  % format URLs
\usepackage{listings}          % format code
\usepackage{enumitem}      % adjust spacing in enums
\usepackage[colorlinks=true,allcolors=blue,breaklinks,draft=false]{hyperref}   % hyperlinks, including DOIs and URLs in bibliography
% known bug: http://tex.stackexchange.com/questions/1522/pdfendlink-ended-up-in-different-nesting-level-than-pdfstartlink
\newcommand{\doi}[1]{doi:~\href{http://dx.doi.org/#1}{\Hurl{#1}}}   % print a hyperlinked DOI

\usepackage{array,multirow,graphicx}
\usepackage[usenames,dvipsnames]{color}
\usepackage[latin1]{inputenc}
\usepackage{tikz}
\usetikzlibrary{shapes,arrows}
\newcommand{\ruzica}[1]{\textcolor{comment_color}{\textsf{RP}: #1}}
\newcommand{\markk}[1]{\textcolor{comment_color}{\textsf{MS}: #1}}
\newcommand{\alex}[1]{\textcolor{comment_color}{\textsf{AR}: #1}}

\newcommand{\lift}{\textit{lift}}



\def\ourTool/{Ives}
\def\lhask/{LiquidHaskell}

\newcommand{\codeinline}[1]{\lstinline[basicstyle=\small]{#1}}



\usepackage{comment}

\usepackage{graphicx}

\definecolor{identifierColor}{rgb}{0.65,0.16,0.16}
\definecolor{comment_color}{rgb}{0.40,0.46,0.3}
\definecolor{num_color}{gray}{0.55}

\lstset{
  basicstyle=\footnotesize,
  breaklines=true,
  frame=bottomline,
  language=haskell,
  %identifierstyle=\color{identifierColor},
  morecomment=[l][\color{comment_color}\ttfamily]{--},
  backgroundcolor=\color{white},   % choose the background color; you must add \usepackage{color} or \usepackage{xcolor}
  breakatwhitespace=false,         % sets if automatic breaks should only happen at whitespace
  %captionpos=b,                    % sets the caption-position to bottom
  %commentstyle=\color{mygreen},    % comment style
  %frame=single,	                   % adds a frame around the code
  keepspaces=true,                 % keeps spaces in text, useful for keeping indentation of code (possibly needs columns=flexible)
  keywordstyle=\color{blue},       % keyword style
  otherkeywords={*,let, Server, Replication, FaultGraph, rankRCG, print, fialProb, goal, ...},           % if you want to add more keywords to the set
  numbers=left,                    % where to put the line-numbers; possible values are (none, left, right)
  numbersep=5pt,                   % how far the line-numbers are from the code
  numberstyle=\tiny\color{num_color}, % the style that is used for the line-numbers
  rulecolor=\color{black},         % if not set, the frame-color may be changed on line-breaks within not-black text (e.g. comments (green here))
  showtabs=false,                  % show tabs within strings adding particular underscores
  stepnumber=1,                    % the step between two line-numbers. If it's 1, each line will be numbered
  stringstyle=\color{mymauve},     % string literal style
  %title=\lstname                  % show the filename of files included with \lstinputlisting; also try caption instead of title
  mathescape=true,
  tabsize=3,
  literate=*{->}{{\textcolor{blue}{$\to$}}}{1}
           {<-}{{\textcolor{blue}{$\leftarrow$}}}{1}
}
  
  
%\usepackage{minted}
%\usepackage{tcolorbox}
%\usepackage{etoolbox}
%\BeforeBeginEnvironment{minted}{\begin{tcolorbox}}%
%\AfterEndEnvironment{minted}{\end{tcolorbox}}%

\begin{document}

\title{Natural Program Synthesis from Examples in Haskell}

%
% any author declaration will be ignored  when using 'pldi' option (for double blind review)
%

\authorinfo{Person 1 \and Person 2}
{\makebox{Department of Computer Science} \\
\makebox{Yale University}  \\
\makebox{A Place, AS 12345}}
{\{person1,person2\}@cs.auniv.edu}


\maketitle

\begin{abstract}
We present a new programming-by-example technique that efficiently synthesizes natural, readable fitting functions that combine user-defined higher-order functions with standard and third-party library code.

The search works by \textit{dismantling} higher-order functions in order to deduce suitable refinement types. These refinement types are then used to prune the search space of possible higher-order functions for a given example set. Since refinement type under-approximate, we can apply Liquid Haskell to arbitrary syntax extensions while still preserving soundness.

We evaluate an implementation of our tool against a large set of synthesis examples including lists, trees, maps, and specialized musical score data structures. This evaluation demonstrates the scalability and versatility of this approach.
\end{abstract}

\section{Introduction}
\label{intro}

Program synthesis is an active research direction~\cite{DBLP:journals/toplas/MannaW80, DBLP:journals/cacm/GulwaniHS12,DBLP:conf/icfp/Bodik15, DBLP:conf/pldi/KuncakMPS10, DBLP:conf/aplas/Solar-Lezama09, DBLP:conf/pldi/SrivastavaGCF11} that aims to automatically derive code from a given specification.
This code is correct by construction and ideally would make a programmer more productive.
Still, writing a complete specification of an entire program is often a more complex task than writing the corresponding code, even for very simple programs.

Programming by example~\cite{cypher93,lieberman01,synasc12} is a promising research direction that enables easy manipulation of data even for non-programmers~\cite{GulwaniHS12}.
Recent work in this area has focused on manipulating fundamental data types such as strings~\cite{flashFillPOPL,vldb12,icml13}, lists~\cite{FeserCD15,Osera:2015} and numbers~\cite{cav12}.
The success and impact of this line of work can be estimated from the fact that some of this technology ships as part of the popular FlashFill feature in Excel 2013~\cite{flashFillPOPL}.

Instead of writing code, the user provides a list of relevant input/output examples and the synthesis tool automatically generates a program that fits.
In this way, the examples can be seen as an easily readable and understandable specification.
However, the code that is produced by these tools is rarely as simple as the specification.
%In~\cite{Osera:2016}, the close connection between refinement types and examples is expounded through the lens of proof theory.
%Other works have explored further ramifications of the theory behind programming by example\cite{Osera:2015, GulwaniHS12, synasc12}.
%These theoretical foundations give us the power and direction to begin to make programming by example a mainstream feature of fully featured languages.

Program synthesis does not inherently address the problem of code readability, often resulting in tools that produce something closer to an executable than the simple and stylistic code a human might write.
Just as with code users write themselves, the ability to reuse and edit synthesized code is an integral part of the programming process.
Thus, we define a notion of \textit{natural} code and define the problem of natural program synthesis.

Take the simple task of synthesizing a list flattening function from examples.
Synthesis approaches that use only the primitive recursive operators~\cite{Osera:2015,FeserCD15}, would find a function similar to \codeinline{solution1} in Listing \ref{natSyn} (modulo language syntax).
Our tool instead focuses on utilizing common library functions, as seen in \codeinline{solution2}.
To quantify our goal of simplicity, we define a notion of \textit{naturalness} of code in Section \ref{sec:naturalness}.
Intuitively, \codeinline{solution2} should have a higher naturalness score than \codeinline{solution1}.


\begin{figure}
  \begin{lstlisting}[caption=Low-level synthesis vs. Natural synthesis,label=natSyn]
    exs = ([[1,2],[3,4]], [1,2,3,4])

    solution1 xs =
      (\c n -> foldr
          (\x b -> foldr c b (id x))
          n xs) (:) []

    solution2 xs = concatMap id xs
    \end{lstlisting}
\end{figure}

\noindent In fact, \codeinline{solution1} is just an application of the GHC\cite{ghc} implementation of concatMap.
To synthesize this solution using only the core higher order functions is certainly motivating, however a user would likely prefer to see a single function, like \codeinline{concatMap}, which exists in a standard library, when using synthesis with the goal of writing their own code.

We choose to focus on functional programming, where a core part of the experience is writing higher order functions~\cite{Lipovaca:2011}.
Library authors often provide users with a number of domain specific higher order functions to enable programmers to more easily write their programs.
Since users write higher order functions with a deep understanding of the domain, using them in synthesis produces code that is more idiomatic and easier to understand then using only the core higher order functions.

In order to facilitate synthesis over user defined higher order functions, we run the synthesis algorithm in two stages.
The first offline preprocessing stage infers rules about how user defined, or imported higher order functions behave over their input and output types.
We encode these rules using refinement types in Haskell with the \lhask/ tool~\cite{DBLP:conf/haskell/VazouSJ14}.
By using all teh information avaible from the examples, we also inferr refinement types on the example set. 
This is used during the online synthesis stage, along with a type match ranking algorithm to efficiently prune and navigate the search space of solution programs.


% fewer examples are needed when leveraging user functions
Although programming by example is an easy entry point for novice users, one of the drawbacks can be the tedious nature of the specification.
For a user, writing out a sufficient number of examples for the synthesis tool to find a solution may involve
  specifying seemingly obvious examples such as \codeinline{[]->[]} in order to cover base cases of recursion.
However, much of this domain specific knowledge is often encoded by the user defined functions, data types and library imports.
By focusing our synthesis procedures on this space, we can reduce the number of required edge case examples and allow users to only provide the important examples.

In Listing \ref{natSyn}, we have already seen the potential to synthesize natural solutions to programming by example queries.
However, a synthesis engine should also be able to synthesize novel (and sometimes unexpected) solutions to problems.
Since the stated goal is to find simple programs that a human might write, this raises the question if finding ``natural'' and and novel programs are at odds with each other.

Our evaluation section shows that this is not the case.
For example, natural synthesis of the Boolean ``or'' function finds two functions, \codeinline{any (id)} and \codeinline{foldr1 (max)}.
The \codeinline{any (id)} solution would expected by Haskell programmer, where \codeinline{any:: (a-> Bool) -> [a] -> Bool)} is a built-in function to Haskell that returns \codeinline{True} if any element of a list satisfies the predicate function.
The more novel solution returned by our natural synthesis is \codeinline{foldr1 (max)}, where \codeinline{max :: Ord a => a -> a -> a} will return the maximum element of the two inputs.
By folding over the list, this solution program exploits the \codeinline{Enum} property of the Boolean type in Haskell in a way that provides insight into some core Haskell functions and types.

We implement this approach in a tool, \ourTool/, that support the real Haskell language, including its native types and any user defined data types.
\ourTool/ synthesizes programs from very few examples to make the specification as simple as possible.
\ourTool/ places a high weight on natural code during the synthesis procedure.
It can also reuse user-defined and library functions in the synthesis procedure, thereby generating code that is both natural to the specific domain.
\ourTool/ is used as a standalone tool as many previous works have functioned, but because it handles proper Haskell, could also be integrated into an IDE.

In summary, we present the following contributions:

\begin{enumerate}
\item A definition of naturalness to describe the complexity of functional code, and a definition of the problem of natural program synthesis.
\item A weighted, type directed, enumeration strategy that generates programs of a high naturalness score for programs that also satisfy user provided examples. 
\item An exploitation of an implicit assumption in existing programming by example work to shrink the search space using refinement types.
\item An evaluation of our implementation, \ourTool/, in Haskell to synthesize programs that utilize Haskell libraries. The tool supports native and user defined data types to produce real Haskell code. The benchmarks show our tool can efficiently generate a wide variety of code that mixes functions from multiple sources.
\end{enumerate}


\section{Motivating Examples} 
\label{examples}

With next few illustrative examples we show functionality of \ourTool/.
It takes as input a list of examples and automatically generates code
corresponding to these examples. The tool is embedded in Haskell, so examples use regular Haskell syntax along with a type annotation.



\subsection{Synthesis with the standard library}

%introduction to us using stutter- from the ground up
We start with a \codeinline{stutter} program. It is one of canonical 
examples used by similar tools \cite{Osera:2016}. The \codeinline{stutter} program takes as input a list of elements and duplicates each element 
of the list.
 
Following the programming by example principles, the user 
provide a list of examples as a list of input/output pairs to the special variable \codeinline{exs}.

\begin{lstlisting}
exs :: [([Int], [Int])]
exs = [([1, 2, 3], [1, 1, 2, 2, 3, 3])]
\end{lstlisting}

Invoking \ourTool/ on this example starts to construct programs  satisfying that when they are applied to the list $[1, 2, 3]$ the result 
is $[1, 1, 2, 2, 3, 3]$.
Using a type-directed enumeration, we look for programs fitting the given example type.
We first search for first order functions of type \codeinline{[Int]->[Int]}.
In addition we look for generalizations of the given type, such as \codeinline{[a]->[a]}, or \codeinline{Num a => [a]->[a]}.
Some functions like this will be \codeinline{cycle} or \codeinline{tail} - which will not satisfy these examples.
Once we have exhausted this class of functions, we begin searching for chains of application.
In particular, we focus on higher order functions because they result in more interesting programs and are more challenging.

One higher order function of interest is in the Haskell the standard library; \codeinline{concatMap :: (a -> [b]) -> [a] -> [b]} which applies a function over a list and concatenates the result.
With \codeinline{concatMap} as a candidate higher order function, we can now search for a component first order function of type \codeinline{a -> [b]}.
We will not find a useful function of such a type, and need to continue searching for generalizations of that type.
We will find the function \codeinline{replicate n :: Int -> a -> [a]} in the standard library, which will replicate an item n times into a list.
This function only needs to be specialized to our examples by applying an initial value.

In the end, \ourTool/ only requires a single example to synthesize the program \codeinline{concatMap (replicate 2)}.
If however, in addition to the above example, the user provided a function \codeinline{dupl} which duplicates an element, there is now a second possible solution to the synthesis.
Rather waiting to find all solutions, \ourTool/ returns solutions to the user as they are found, then proceeds with rest of the search.
For this example, \ourTool/ will return the solution \codeinline{concatMap dupl} after X seconds, continue searching and return \codeinline{concatMap (replicate 2)} after X+Y seconds.

\begin{lstlisting}
dupl :: a -> [a]
dupl x = [x,x]
\end{lstlisting}

In order to return better solutions in the beginning of the search, we use a ranking system based on the complexity of the generated programs.
We introduce a new complexity measure called \textit{naturalness} to help guide our search.
Naturalness will also help us avoid the difficult to read solutions as those often seen in other programming-by-example systems.

If programming-by-example is to be integrated into mainstream programming environments, the generated code must be readable and editable.
The code should also support a real language and use native datatypes to the language.
While in MYTH~\cite{Osera:2015} the focus was on lists as inductively defined data types, we are focusing on using the built in representation of a list in Haskell.

\subsection{Optimizing with types}

If in our search, we look for every function of the correct type, we quickly explore the search space.
By using information embedded in the examples, we can infer refinement types to help prune the space.
Take the example of synthesizing a function that takes the odd numbers from a list.
Again, only a single example is needed for \ourTool/ to find a simple solution program.

\begin{lstlisting}
exs :: [([Int], [Int])]
exs = [([1, 2, 3], [1,3])]
\end{lstlisting}

Rather then beginning a search over all functions of type \codeinline{[Int] -> [Int]}, we can first infer a refinement type.
This refinement type will specify that the input must be at longer than the output.
\ourTool/ then searches for a higher order function consistent with this refinement type.
For simplicity, imagine we only have the higher order functions \codeinline{map} and \codeinline{filter} available.
\codeinline{map} will be assigned a refinement type that specifies the input and output lengths are equal - however is pruned from the search since this observation is not consistent with the example's refinement type.
\codeinline{filter} will be assigned a refinement type that specifies the input length is greater than or equal to the output length.
This observation is consistent with the examples, and is thus used to synthesize the solution \codeinline{filter (odd)}.


\subsection{Synthesis with user defined values}

Working on a set of user defined code is also a critical task \ourTool/ supports. 
In the next example the user has provided a binary tree data structure and a higher order function to map over it. We show the synthesis of the exceedingly (for the sake of brevity) simple program \codeinline{mapBTree not}.
Doing such synthesis requires automatic reasoning about not only the user defined polymorphic data type, but also the higher order function they have defined over it.

\begin{lstlisting}
data BTree a = Nil |
               Branch (BTree a) a (BTree a)

mapBTree :: (a -> a) -> BTree a -> BTree 
mapBTree f Nil = Nil
mapBTree f (Branch b1 v b2) = 
  Branch (mapBTree f b1) (f v) (mapBTree f b2)

exs :: [(BTree Bool, Tree Bool)]
exs = [(Branch Nil True Nil,
       Branch Nil False Nil)]
\end{lstlisting}

It may seem that if a user can write a higher order functions over custom data structures, they would not have a need to synthesize such functions.
However, imagine the case of a user importing libraries.
Haskell's module system and large repository of libraries like Hackage and Stackage are an indispensable part of the language\cite{hackage,stackage}.
Often, a user is importing a library that is large, unfamiliar, and/or poorly documented.
Using \ourTool/, the user no longer needs an intimate knowledge of the library to makes use of the functions and datatypes, and can instead synthesize functions from examples.


\subsection{Synthesis with a DSL}

As an example, we show code to transpose a music value from the Euterpea DSL (domain specific library) for music\cite{euterpea}.
Among other things, Euterpea defines a tree-like datatype called \codeinline{Music} and various functions for manipulating these types.
The user only needs to express the basic datatype as examples, and \ourTool/ can synthesize the \codeinline{solution} program.
The solution utilizes the functions from Euterpea; \codeinline{mMap} for mapping over music values, and \codeinline{(trans::Int -> Music Pitch->Music Pitch)} to transpose a Music Pitch by a value.
This again requires automated reasoning about the properties of the library's data types and higher order functions.
Because we have synthesized a natural looking program, the user does not need to understand details of the library's function and data structures to be able to immediately gain an intuition about how the solution program works.

\begin{lstlisting}
import Euterpea

exs :: [(Music Pitch, Music Pitch)]
exs = [
  (Prim (Note qn (C,4)):+:Prim (Note qn (D,4)),
   Prim (Note qn (D,4)):+:Prim (Note qn (E,4))) ]
        
solution = mMap (trans 2)
\end{lstlisting}



\section{Problem Formulation}
\label{problem}

\subsection{Naturalness}

The goal in natural synthesis is synthesize programs that meet a specification with the simplest and most readable code.
For programming by example, the specification is a set, $Ex$, of pairs of input and output $(i,o)$.
As a formalization of simplicity and readability, we present a definition of naturalness.

The standard of measuring complexity (the inverse of naturalness) is cyclomatic complexity\cite{cyclo}.
Cyclomatic complexity is function, $\mathcal{CC}$on the control flow graph of a section of code, $C$.
The function is $\mathcal{CC}(C) = E − N + 2(P)$, where $E$, $N$, and $P$ are the edges, nodes and connected components respectively.
This function measures "the number of linearly independent paths" through a program, a crucial part of manipulating stateful variables.
However a this is not well-suited for pure functional languages that use branching in a more functional way.
For example, given two programs to compute absolute value, the "if" solution should be more natural, but using cycolmatic complexity verbatim means
\codeinline{f x = if x>=0 then x else x*-1} is more complex
\codeinline{f x = x * (fromEnum (x<0) * (-1) + fromEnum (x>0))}

Instead, we define naturalness to be the number of nodes in the abstract syntax tree.
This approach integrates cyclomatic complexity, because branching is still encoded as a measure of complexity.

%Our definition of naturalness matches well many techniques valued in functional community.
Currying fits under this definition of naturalness.
\codeinline{f = (1+)} is more natural than \codeinline{f x= 1 + x}.
This also favors predefined functions over generating new lambda terms.
Taking the code from Listing \ref{natSyn}, $\mathcal{N}($solution1$)$ has a naturalness score of 1/8, while $\mathcal{N}($solution2$)$ has a score of 1/3.

The problem is then to find an expression e, such that $\forall (i,o) \in Ex, e (i) = o \land max(\mathcal{N}(e))$.


\subsection{Enumeration}
In order to solve the above problem, we will immediately restrict our search space to exclude generated lambda terms, as such terms will generally induce a very low naturalness score and explode the search space.
Our synthesis approach will then only be able to solve synthesis problems when a solution exists that only draws from a finite set of predefined expressions.

Let $E$ be the finite set of expressions exposed to the top-level module from user code, imported libraries, and the core library.
Our search space will be the set of permutations of well-typed applications of elements of $E$.

To determine if an expression is well-typed, let $T$ be the set of types (both inferred and explicitly declared) exposed to the top-level module from user code, imported libraries, and the core library.
A type environment $\Gamma$, is the set $\{e1 : \tau_1,\ ...,\ e_n : \tau_n\}$, where $e_{i} \in E$ and is of type $\tau_i \in T$.
The set of well-typed expressions we consider, $\mathcal{G}$, is the infinite set $\{e1 : \tau_1,\ ...,\ e_n : \tau_n\}$, where $e_i : \tau_i$ follows the usual rules of application for constructing well-typed expressions from $\Gamma$. 

We place a constraint on the example set that $Ex:\{(\tau_i,\tau_o)\}$, or more specifically $\forall (i,o) \in Ex,\ i:(\tau_i \in T) \land o:(\tau_o \in T)$.
In practice this a trivial constraint, achieved by requiring the user to provide the types explicitly \cite{Osera:2015} or to infer the types \cite{gulwani_popl15} based on regular expressions.



\subsection{Lifting Example Types}
While conceptually simple, enumerating all well-typed functions is wasteful - if possible we need to prune the search space.
To do this, we can exploit an unstated assumption, but widely accepted approach, in existing programming-by-example work.
Usually, the example pair type is lifted into a function type in the trivial way.

\begin{flalign*}
lift\ (\tau_i, \tau_o) =\ \tau_i \to \tau_o
\end{flalign*}

However, a subtyping relation can create more specific types that will better prune the space.
The subtyping relation $A<:B$ means that any time type $B$ results in a well-typed program, so would type $A$ in place of $B$.
A subtyping relation induces a subset relation of terms of type $A$ in relation to the terms of type $B$.
Given $\mathcal{A} = \{ x | x::A\}$ and $\mathcal{B} = \{ x | x::B\}$, we have $\mathcal{A}\subseteq\mathcal{B}$.
Lifting the examples to a subtype of the trivial lifting can then yield a smaller search space.

\begin{flalign*}
lift'\ (\tau_i, \tau_o) <:\ \tau_i \to \tau_o\\
\end{flalign*}

Notice that we did not write out a full function for the subtype.
This would have implied a subtyping on the component types, specifically the inputs and outputs would be contravariant or covariant, respectively.
However, we do not wish to restrict the domain or range of the function, but only the size of function space.
So we must have the following

\begin{flalign*}
lift'\ (\tau_i, \tau_o) =&\ \tau^{s}_{i} \to \tau^{s}_{i} \nRightarrow\\
(\tau^{s}_{i} <: \tau_i)\ \lor&\ (\tau_o <: \tau^{s}_{o})
\end{flalign*}


As an demonstration of this approach, following the syntax from previous code samples, we demonstrate below the synthesis of \codeinline{map (+1)}. We provide examples of type \codeinline{([Int],[Int])}.
\begin{lstlisting}
exs :: [ [Int] :-> [Int] ]
exs = [
  [1]   :-> [2],
  [3,4] :-> [4,5] ]
\end{lstlisting}

Using the traditional approach, we would have the trivial lifting to the function type.
However, if we use $lift'$ instead, we would derive a more specific type.
This more specific type could be a refinement type, expressed here using the syntax of LiquidHaskell\cite{DBLP:conf/icfp/VazouSJVJ14}.
 
\begin{lstlisting}
([Int],[Int]) = [Int] -> [Int]}\\
\{x:[Int] -> y:[Int] | length x > length y\}}
\end{lstlisting}

Alternatively, we could also have derived the equally specific type using dependant types\cite{dependant_types}.
In this case, we would need a definition of (\codeinline{Vec L a}) to describe the type of lists of length \codeinline{L} and elemental type \codeinline{a}.

\codeinline{\{Vec L Int -> Vec L Int\}}

Notice that in either case, the target function type is a subtype of the trivial lifting, but we have not changed either the types of either the domain or range of the target function.

Our approach to natural synthesis is then: given a type environment $\Gamma$ and an example set $Ex:\{(\tau_i,\tau_o)\}$, enumerate $\mathcal{G}$ in order of naturalness, such that $e : lift'(\tau_i,\tau_o)$.
This list can then be checked in order to find an $e$ such that $\forall (i,o) \in Ex, e (i) = o$.

\subsection{function classification}
we need to assign all functions in $\mathcal{G}$ a refinement type in order to use the above result.
Hence we do offline analysis.



%
%This framing will draw the type of the target expression directly from the examples.

\subsection{Solution Space}\label{solnSpace}
While this approach can work for first-order synthesis, we instead focus on data structure manipulation problems that can be solved with higher order functions.

We require all higher order functions to be of a unified signature \texttt{$\_ \to * \to *$}, where the final kind of the signature is a function mapping the input type to the output type. 
Here, a kind is understood to be the type of a type constructor, in this case \texttt{$\to$}, which constructs a function type from two other types.

The practical consequence of this format is that a user must partially uncurry (collapsing trailing function arguments into a single tuple argument) any higher-order function they are interested in using during synthesis.
This also means that any type variable appearing in the higher-order function must be accounted for in the input and output types so that all type variables in its signature can be resolved.
This allows us to conclude that any types that are between the input and first order function will be static initial values, which can be assigned using the process described in Section \ref{makeFxns}.
This is a simple procedure that makes use of the user's domain knowledge of which parameters to the function will be given by the examples; consider:

\begin{lstlisting}
zipWith' :: (a -> b -> c) -> ([a], [b]) -> [c]
zipWith' f (xs,ys) = zipWith f xs ys
\end{lstlisting}

By formally defining the space of functions we are interested in synthesizing, we can this definition to prove some properties on the algorithm.
In particular we show in Section \ref{sound} that \ourTool/ is complete for this subset of functions.

the solutions \ourTool/ supports synthesizing are higher-order data structure manipulation programs.
The higher-order functions take a component function that is a first-order function, for example \codeinline{(+)}.
The solution programs can be expressed as:
% up to reordering of terms (we dont actually support this, should we really include this)

\begin{lstlisting}
solution ::
           (* -> types)  -- Component Function
        ->  types        -- Initial Values
        ->  *            -- Input
        ->  *            -- Output
types = * | * -> types
-- * matches on type variables and constructors.
\end{lstlisting}

Generally, the component function is applied across the \textsf{input} data structure, which the \textsf{solution} uses to construct an \textsf{output} data structure or reduction. As we will argue in Section \ref{evaluation} this set is expressive enough to support the classic \texttt{map}, \texttt{filter}, and \texttt{fold} functions, as well as higher order functions found in imported modules and user-supplied code.

Our goal is to create a synthesis procedure that is easily portable across full implementations of functional languages (Haskell, OCaml, etc), so we prefer using a type directed approach to synthesis over explicit code analysis whenever possible. This increases the portability and longevity of our system. For this implementation we target Haskell, detailing the exact modifications needed to expand this to other languages in Section \ref{languageSupport}.

%Our algorithm does not explicitly try to fit component functions to the examples. Instead, we leverage a promising body of existing work in synthesizing top-level, first-order functions \cite{potential, reviewers}. While it is out of scope to go in to detail, we will briefly discuss the integration of these synthesis procedures in Section \ref{conclusions}.

%The liquidHaskell predicate applied to this signature will be of the effect of \texttt{len([a],[b]) = len([c])}.


\section{System Overview}



\begin{figure}[t]
  \centering
% Define block styles
    \tikzstyle{block} = [rectangle, draw, fill=none, 
    text centered, sharp corners, minimum height=3em]
\tikzstyle{line} = [draw, -latex']
    
\begin{tikzpicture}[node distance = 8em, auto]
    % Place nodes
    \draw [fill=gray!20, opacity=1] (-4,-5) rectangle (4,-3);
    \node [block](hofu) {select HO functions};
    \node [block, above left of=hofu] (libraries) {standard libraries};
    \node [block, above right of=hofu] (API) {user definded functions};
    \node [block, below of=hofu, node distance=6em] (refty) {assigning refinement types};
    \node [block, below of=refty, node distance=6em] (engine) {synthesis engine};
    \node [block, left of=engine, node distance=8em] (examples) {examples};
    \node [block, right of=engine, node distance=8em] (program) {program};

%Online Flow

    % Draw edges
    \path [line] (libraries) -- (hofu);
    \path [line] (API) -- (hofu);
    \path [line] (hofu) -- (refty);
    \path [line] (refty) -- (engine);
    \path [line] (examples) -- (engine);
    \path [line] (engine) -- (program);

\end{tikzpicture}
  \caption{High-level structure of the algorithm.}
  \label{fig:high_level_overview}
\end{figure}

Figure~\ref{fig:high_level_overview} gives a high-level description of ways in which the components of our algorithm interact. Broadly speaking, there are two main stages in the algorithm. The offline (preprocessing) phase gathers the higher order declarations visible in the APIs and user-provided code, and assigns refinement types to them to build a custom synthesis \textit{engine}. This engine is then used during the online phase of the algorithm to search for functions that fit a set of supplied examples.

During the offline phase, the algorithm first scans the user-provided code, the libraries it imports, and the standard library to gather all of the functions and global values visible to the program. Then, it selects the higher-order functions from the set of all functions and values, and uses Liquid Haskell \cite{DBLP:conf/haskell/VazouSJ14, DBLP:conf/esop/VazouRJ13, DBLP:conf/icfp/VazouSJVJ14} to assign refinement types to each one. Finally, each higher-order function is assigned a weight based on locality\cite{DBLP:conf/pldi/GveroKKP13}. User-defined functions are given the highest priority, while direct imports are given less, and the standard libraries are given the least. Together with the first-order functions and values, these triples of higher-order functions with their refinement types and weights are collected to produce a synthesis engine.

Once this stage is complete, the user can examples to the synthesis engine, which will search the space of constructable functions for those that fit the examples. First, the engine computes a refinement type that fits the examples. This type is matched against the refinement types of the known higher-order functions, and the weights of each known function are adjusted based on how close the types match, if at all.

Once the candidate higher order functions have been chosen, the synthesis engine performs a best-first search for a program that fits all of the input and output examples by composing the candidates with first-order functions. For example, the higher-order function \texttt{map} might be supplied the \texttt{length} function if the example inputs are lists of lists of integers and the output examples are all lists of integers. The programs that are examined during the search are evaluated against the example set and are reported to the user as they match. Because the weights favor local declarations, the highest-ranked programs are likely to be the most idiomatic.

We present a pseudocode algorithm here which we will use as a roadmap for the rest of the paper, explaining each line in the proceeding sections. For the remainder of the paper, code samples are taken from the implementation, and modified slightly to elide the details of managing Haskell's type system.
 
\begin{lstlisting}[caption=A pseudocode representation of the build and synthesis stages of the synthesis algorithm, label=listing:Algo]
main = do
  eng <- build
  ex  <- getExamples
  synth eng ex
  
build = do
  allTypes   <- collectTypesAndWeights
  allHOTypes <- filter isHigherOrder allTypes
  allRTypes  <- assignRTypes allHOTypes
  return (allTypes,allRTypes)
  
synth eng ex = do
  -- assign refinement types to examples
  exType   <- getExampleType ex
  exRType  <- assignRTypes exType
  -- make candidate functions and programs
  hoFxns <- rankByTypeMatch exRType eng
  progs  <- makeFxns exType hoFxns
  -- test the ranked list of possible programs
  validProgs <- filter (testOn ex) progs
\end{lstlisting}


\section{Offline: Synthesis Engine Construction}


\subsection{Refinement types for higher order functions}\label{HORtypeInf}

% collectTypesAndWeights
The first step of our algorithm (line 7 of Listing \ref{listing:Algo}) is to collect all of the type signatures from our sources (user code, imports, and standard library). In order to rank the higher-order functions, we assign weights based on their source location. User-defined functions are given the highest priority, while direct imports are given less, and the standard libraries are given the least. These rankings will contribute to the final ranking of candidate functions in the synthesis stage when we match the component function signatures on the examples.

% filter isHigherOrder
We filter through these to select only the higher order functions. Because in Haskell the function type constructor ($\to$) is right binding, any higher order functions must have parenthesis in the type signature, which provides a convenient filtering predicate. The is over approximating filter, since the type signature might contain extraneous parentheses, for example surrounding the entire signature. In practice, it is rarely the case that a programmer will add extraneous parentheses to type signatures and this does not significantly impact performance.

% assignRTypes
In line 9 of Listing \ref{listing:Algo} we call the \codeinline{assignRTypes} function (shown in Listing \ref{listing:addRType}) to automatically generate refinement types that relate input and output sizes for our higher order functions.
These refinement will be used to prune the search space in synthesis, as explained in detail in Section \ref{synth}.
In brief, the size relation that applies to the higher order functions must also apply to the examples in order to consider that function as a candidate.
When writing the refinement types, we can be sure, by our specification in Section \ref{problem}, that the last two types are always the input and output.
We can then make use of type holes, in order to account for the diversity of component functions and initial values that might be required for any given higher order function.

\begin{lstlisting}[numbers=none]
map :: _ $\to$ i:[a] $\to$ {o:[b] | (len i) = (len o)}
\end{lstlisting}

%> map rTypeTemplate ["=","<=",">="]

For every predicate we test against, we are able to more accurately prune the search space of higher-order functions.
However, since we must test many higher-order functions on each these predicates, the cost to add a predicate is high.
Therefore, it is best to only select as many refinement types as are needed.
We only use predicates of $\leq,=,\geq$ to specify size constraints on input and output.
Notice that map will actually satisfy all three of these predicates, which in general, results in an over approximation of appropriate refinement types for higher order functions.
We discuss potential optimizations for this in Section \ref{evaluation}.

\begin{lstlisting}[caption=Adding refinement types to higher order functions,label=listing:addRType]
assignRTypes ::Sig -> IO(Sig, [RType])
assignRTypes sig = do
   x <- if eqTypes (lastTypes sig) 
       then rTypeAssign sig
       else return [noRType]
   return (t, x)

testRs :: Sig -> IO([RType])
testRs s =
  filterM (runLiquidHaskell s) allPossibleRTypes
\end{lstlisting}

We separate possible types into two cases using the \codeinline{eqTypes} function on line 3 of Listing \ref{listing:addRType}.
In the case that input and output types (extracted with \codeinline{lastTypes sig}) of the higher order functions are the same (up to equality on the top level type constructor), we should generate refinement types.
In the other case, when the input and output type are different, the size measures between two different type constructors are not guaranteed to have any significance.
A relation on these values may be useful on occasion, but in practice is more often only a confounding factor, leading to wasted computation.
When we do not assign refinement types to a higher order function, we tag the higher order function with the placeholder \codeinline{noRType} value.
These \codeinline{noRType} tagged functions can be further pruned in the synthesis stage by utilizing a subtype ranking system to be explained in more detail in Section \ref{synth}.

\subsection{User defined data types}
%We focus only higher order functions that manipulate data structures
In order to support user defined data structures, we only require that a user implements some kind of measure\cite{DBLP:conf/haskell/VazouSJ14} over their data structure.
This size function will allow \lhask/ to determine size constraints on the examples, so that \ourTool/ can pick higher order functions that also satisfy those size predicates.
In fact, the size function could just be a constant function, resulting in every function and example satisfying the equality refinement type predicate.
This means the system will test every higher-order function that fits the types.

As an example, take the code from Section \ref{examples} for synthesizing a music function.
The user would have needed to provide a measure function for Music a.
This measure will allow \lhask/ to draw conclusions about the size of examples of type \codeinline{[Music a :-> Music a]}, as well as conclusions about higher order functions over the Music data structure.
In this context, one sensible measure function counts the number of notes (Prim) in the tree-like Music structure, as shown by the \codeinline{len} function in Listing \ref{lenFxn}.

\begin{lstlisting}[caption=a user defined measure over a datatype,label=lenFxn]
import Euterpea

{-@ measure len @-}
len :: Music a -> Int
len m =
  case m of
    Prim _     -> 1
    m1 :+: m2  -> len m1 + len m2
    m1 :=: m2  -> len m1 + len m2
    Modify c m -> len m
\end{lstlisting}





\section{Online: Fitting Functions to Examples} \label{synth}

With the synthesis engine constructed, the system is ready to synthesize programs from examples.
Multiple programming-by-example queries can then be answered using this engine.
The synthesis engine only needs to be reconstructed when there are new library imports, or when there is a revision of the user-supplied code.

When examples are provided, the synthesis engine finds a suitable refinement type for a hypothetical function that could fit that example.
Then, \ourTool/ filters and ranks the higher order functions based on the refinement types known to the engine and the example types provided.
Once the candidate higher functions are identified, \ourTool/ will select and build first order functions that match the type of the higher order function's component signature to build a final set of candidate programs.

Each of these candidate programs is executed in best-first order against the set of inputs.
Whenever a function produces the correct outputs for each input, it is said to fit, and is reported to the user.
This search continues until the space is exhausted or it is manually interrupted.
The search will always terminate since we are working over a finite space of generated functions, are our type reductions are strictly decreasing, which we will explain in Section \ref{typeMatch}

% getExampleType
% assignRType
\subsection{Refinement types for examples}
% getExampleType
As in Section \ref{HORtypeInf}, we also consider two cases for examples. The first, where the example input and output types match up to the top level type constructor, and the the case where the types do not match.

% assignRType
In the case that the types do match, we find the set of refinement types that the examples satisfy. Generating refinement type predicates about the size of the input and output, as in Section \ref{HORtypeInf}, we apply the same algorithm from Listing \ref{listing:addRType}. 
For instance, an example set for \codeinline{filter (>3)} might look as follows:

\begin{lstlisting}[caption=Refinement type inference for examples,label=exRTypeGen]
exs :: ([Int] , [Int])
exs = [([1,2,3] , [1,2,3]),
      ([1,3,4] , [1,3]),
      ([4,6,8] , [])]
       
exsRType ::
  (i : [Int], { o : [Int] |
  len i >= len o })
\end{lstlisting}

\noindent and have the final refinement type of \codeinline{exRType}, since all of the examples suggest that the output list does not grow. 
Again, when the types do not match we assign the \codeinline{noRType} flag to the examples, as we did for higher order functions in Listing \ref{listing:addRType}.
We can now reduce our search space to only higher order functions with the same refinement type that matches the examples' refinement type. 


\subsection{Type match ranking}\label{typeMatch}

Once \ourTool/ has both the base and refinement types for the examples and higher order functions, it can can prune and order this set (line 17 of Listing \ref{listing:Algo}).
The first step is to simply filter the higher order function candidates over equality of refinement types.
Additionally, \ourTool/ will check the example types are concrete versions of the input/output types of the higher order function with the infix (for clarity) \codeinline{isConcreteTypeOf} function.
For type A to be a concrete version of type B, there must exist some type C (possibly equal to type B), such that both A and B can be instantiated to that type.
The above requirement is then that there is some way to unify these two types - a familiar problem\cite{typeUnif}.

\begin{lstlisting}[caption=Pruning based on types]
filter (exRType ==) higherOrderRTypes
filter (exType `isConcreteTypeOf') higherOrderComponentTypes
\end{lstlisting}

Once these higher order functions have been culled from the pool of candidates, we update their ranks that had been assigned in Section \ref{HORtypeInf} from code locality.
The higher order function can advance in the ranking by using a value function to find out exactly how much the example type \codeinline{isConcreteTypeOf} to the input/output types of candidate higher order function.

In Listing \ref{valueAlgo}, we present a demonstration of part of this ranking algorithm.
As we traverse the tree structure of the type, the more pieces of the type signature that match, the higher the value of that match. 
However, if there is a type constructor mismatch, the two types can never be reconciled, and the entire value gets nothing.

\begin{lstlisting}[caption=Type closeness ranking algorithm (sample),label=valueAlgo]
value :: Type -> Type -> Maybe Int
value (TyFun i1 o1) (TyFun i2 o2) = fmap (1+) 
    (liftA2 (+) (value i1 i2) (value o1 o2))
value (TyCon n1) (TyCon n2) =
   if (n1==n2) then Just 20 else Nothing
value (TyCon n1) (TyVar _) = Just 10
value _ _ = Nothing
\end{lstlisting}

As an example of how this value function is applied the higher order functions, imagine we have three map functions specialized on particular values. 
The fully polymorphic map will score 1 point for having a function between input and out, 2 points for both having lists, and 20 points for a type variables matching a type constructor, for a total of 5 points. The mapI for Ints, will score the same, but score 20 points for each matching type constructors instead of 10 points for each type variable matched to a type constructor. The mapB for Boolean value gets nothing since there is no way to reconcile that type to the example type.

\begin{lstlisting}[caption=Ranking higher order function,label=horank]
exs ::                   ([Int] , [Int])
map  :: (a    -> b)    -> [a]    -> [b]
mapI :: (Int  -> Int)  -> [Int]  -> [Int]
mapB :: (Bool -> Bool) -> [Bool] -> [Bool]

-- map  scores 5
-- mapI scores 43
-- mapB scores Nothing
\end{lstlisting}


\subsection{Component function generation}\label{makeFxns}
% makeFxns

\ourTool/ must now find first order function for each of the higher order functions that are still candidates (line 18 of Listing \ref{listing:Algo}), in order to create complete functions in $\mathcal{G}_I$.
For a given higher order function, \ourTool/ can choose component functions by reusing the weighted type matching algorithm from Listing \ref{valueAlgo}.
Since examples must be given as a concrete type, we can always partially specialize our candidate higher order function. 
We then search for first order functions that will type check against the partially specialized component signature.
This partial specialization is a way of extracting more information out of our examples, and significantly reduces the space of candidate first order functions.
Similar to Listing \ref{horank}, we show an example of how type matching is applied over first order functions in Listing \ref{comprank}.

\begin{lstlisting}[caption=Ranking component function,label=comprank]
exs   ::                ([Int] ,  [Int])
map   :: (a   -> b)   -> [a]   -> [b]
mapEx :: (Int -> Int) -> [Int] -> [Int]

component ::
      Int    -> Int
f1 :: a      -> b      -- value is 21
f2 :: Int    -> a      -- value is 31
f3 :: Int    -> Int    -- value is 41
f4 :: [Bool] -> [Bool] -- value is Nothing
\end{lstlisting}

\subsection{Initial Values}\label{initVals}

In addition to considering first order functions where the arity of the type signature is equal to the component function, we may also want ``larger'' functions that have been applied to initial values.
For examples, if the component signature is \codeinline{::Int->Int}, we may have the first order functions \codeinline{(+)::Int->Int->Int} in scope.
By applying an initial value to \codeinline{(+)}, we can get a new function (e.g. \codeinline{(+1)::Int->Int}) that fits the component signature.

In order to use an initial value, it must be in scope, i.e. in the set $\Gamma$.
Initial values can be placed into this set in a few ways.
First, users may have some domain knowledge that a particular value, or set of values, may be useful in their application.
In this case, the value only needs to be defined in the same file as the examples.
Users may also write their own specializations of the higher order functions if the value should only be used in the context of a single function.

This approach also handles the case when higher order functions to need initial values in addition to a component function.
For example, the \codeinline{map} function only takes a first order function, while \codeinline{foldl :: (a-> b-> a)-> a-> [b]-> a} requires an initial value for \codeinline{a}.
Using a similar process as for first order function application, we can apply values until the higher order function only needs the example input to complete execution.
To identify initial values in a higher order type signature, we can use our previous assumption that all higher order function have been partially curried to the type \codeinline{_ -> *-> *}. 
Adding the further assumption that only one first order function maybe be passed to the higher order function, we simply tag any non-function type in the hole as an initial value.

\begin{lstlisting}[caption=adding default initial values]
-- to use 5 as an initial value for foldl
fold5 :: (a -> b -> a) -> [b] -> a
fold5 f i o = foldl f 5 i o

-- to use 5 as an initial value for all functions
initVal :: Int
initVal = 5
\end{lstlisting}

Another interesting way to obtain initial values is through the Monoid class.
If the initial value's type is an instance of Monoid, then the identifier mempty, from Haskell's monoid typeclass\cite{monoid}, will be included in $\Gamma$.
As an example, importing the Data.Monoid library, will bring into scope the unit element is for lists, \codeinline{mempty= []}.
There is no Monoid instance for Int in Data.Monoid however. 
There are two valid monoids for numbers, using either (+) or (*) as the operators and resulting in unit elements 0 and 1 respectively. 
Because these are particularly common value,  we hard code both of these values (along with the other useful values of -1, and 2), bringing them into $\Gamma$ and thereby making them available by default in synthesis.

Presented with the problem of finding integer values to satisfy the examples may initially seem like a good application for an SMT solver.
However, keep in mind that we do not in general know what we are trying to solve - the actual use of these variables is hidden within the function definition.
Since in this work we maintain a primarily type directed approach, rather than code analysis, we will not be able to unravel these functions.




\section{Evaluation}\label{evaluation}
\section{Evaluation}\label{evaluation}

\subsection{Soundness and Completeness}\label{sound}

\markk{TODO THIS SUBSECTION IS ALL JUNK NOW WITH NEW FORMALISM}
It is clear that no function will be returned by the algorithm that does not fit the examples given, since functions are validated before being reported.
Therefore, it is trivial to conclude that \ourTool/ is sound over the given examples.
Still, it is possible for the synthesis procedure to return a function that does not capture the user's intent - that is, as with any programming by examples system, \ourTool/ is not sound over the user intent.
Generally, this ambiguity can be resolved by the user supplying more examples to narrow the set of possible fitting functions.
However, depending on what the user is trying to synthesize, and which examples have been provided, it is possible for new examples to increase the internal search space.
If, for example, a user gives only positive examples for a \texttt{filter}, the refinement type predicate discovery will assume that the lists do not change size, and will likely return \texttt{map id} as a result.

The completeness claim we might like to make is that over the solution space defined in Section \ref{problem}, we will always find a solution if it exists.
Since our space is finite (TODO considering only initial values induced by the types monoid instance), completeness can be made trivially true by replacing all instances of pruning with a zero ranking, so that our algorithm now is only a best-first enumerative search.
Because we make some decisions in pruning that removes potentially sound functions, such as using the \codeinline{noRType} tag in Section \ref{HORtypeInf} we trade this completeness for performance.
In Section \ref{sec:related}, we will discuss why, even if we had completeness, it should be sacrificed in future work.
%On the other hand, the set of functions that the algorithm can produce is fairly broad. It is able to search through the entire space of higher order functions that have been specialized with a first-order function, when considering the functions that are in scope. We will see in Section \ref{evaluation} how broad this space actually is. \markk{See Section \ref{solnSpace}}


\subsection{Performance}

\begin{table*}[t]
  \centering
  \bgroup
  \def\arraystretch{1.1}
  \begin{tabular}{|c|l|c|l|l|l|l|}
    \hline
    & Name & Time (s) & Imports & \# Ex & Representative Example & Generated Function \\
    \hline
    \parbox[t]{2mm}{\multirow{4}{*}{\rotatebox[origin=c]{90}{Bool}}}
    & and & 2.02 & None & 3 & [True, False] $:\to$ False & all id \\
    & and-2 & 5.52 & None & 3 & [True, False] $:\to$ False & foldl min True \\
    & or  & 3.95 & None & 4 & [True, False] $:\to$ True & any id \\
    & xor & 5.59 & None & 4 & [True, False, True] $:\to$ False & foldl xor False \\
    \hline

    \parbox[t]{2mm}{\multirow{4}{*}{\rotatebox[origin=c]{90}{Tree (u.d.)}}}
    & double vals & 3.35 & None & 1 & ((1) 3 (2)) $:\to$ ((2) 6 (4)) & mapBTree (*2) \\
    & tree id & 2.49 & None & 1 & ((1) 3 ((4) 5 (6))) $:\to$ ((1) 3 ((4) 5 (6))) & mapBTree id \\
    & tree max & 2.95 & None & 3 & ((1 10) 5) $:\to$ 10 & accumTree max 0 \\
    & tree sum & 2.93 & None & 1 & ((3 1) 2) $:\to$ 6 & accumTree (+) 0 \\
    \hline

    \parbox[t]{2mm}{\multirow{9}{*}{\rotatebox[origin=c]{90}{List}}}
    & all even & 2.02 & Data.List & 4 & [2,4,6,8] $:\to$ True & all even \\
    & some odd & 4.70 & Data.List & 3 & [1,4,5,6] $:\to$ True & any odd \\
    & custom filter & 11.88 & Data.List & 3 & [1,2,3,4,5] $:\to$ [3,4,5] & filter user\_pred \\
    & length & 1.20 & Data.List & 3 & [5,6,7,8] $:\to$ 4 & foldl count 0 \\
    & max elem & 2.91 & Data.List & 3 & [4,10,7] $:\to$ 10 & foldl max 0 \\
    & negate all & 7.48 & Data.List & 1 & [True, False, True] $:\to$ [False, True, False] & map not \\
    & odd prefix & 8.77 & Data.List & 1 & [1,3,4,6,7] $:\to$ [1,3] & takeWhile odd \\
    & stutter & 3.02 & Data.List & 1 & [1,2,3] $:\to$ [1,1,2,2,3,3] & concatMap (replicate 2) \\
    & sum ints & 4.64 & Data.List & 1 & [1,2,3,4] $:\to$ 10 & foldl (+) 0 \\
    \hline

    \parbox[t]{2mm}{\multirow{4}{*}{\rotatebox[origin=c]{90}{Etc.}}}
    & set sum & 1.83 & Data.Map & 1 & \{ 1, 2, 3, 4 \} $:\to$ 10 & Data.Map.foldl (+) 0 \\
    & music id & 7.47 & Euterpea & 1 & C\# $:\to$ C\# & mMap id \\
    & transpose score & 5.15 & Euterpea & 1 & A $:\to$ B & mMap (trans 2) \\
    \hline
  \end{tabular}
  \egroup
  \caption{Benchmarks and Performance Measures. This table lists all 20 benchmarks, grouped by data structure. Each benchmark lists its name, the amount of time it took to synthesize, the extra imports it uses, the number of examples needed to synthesize, one representative example, and the synthesized function itself. The group marked ``Tree (u.d.)'' is a user-defined structure with user-defined higher-order operations. All reported data is generated on a Linux machine with four cores of Intel i5-3450 @ 3.10GHz and 8 Gb of ram.}
  \label{tab:benchmarks}
\end{table*}

In Table \ref{tab:benchmarks} we show detailed information about \ourTool/ over a set of benchmarks. These benchmarks were chosen to show the versatility of \ourTool/ over many different applications and libraries. The benchmarks over booleans, trees, and lists are common to many other programming-by-example tools. The examples that utilize the \codeinline{Data.List} and \codeinline{Euterpea} libraries to show \ourTool/'s ability to work with large, highly specialized, 3rd-party libraries. Due to the algorithm's focus on generating natural code, the synthesized functions are concise enough to be listed within the table itself. The representative examples show that few, simple hints to the synthesizer are able to produce good results. In many cases, the representative examples are actually the \textit{only} examples necessary to synthesize the desired function. This shows that our approach uses the information available to it effectively.

In addition, the runtime average about ten seconds thanks to the inherently parallel nature of the search. With just a few lines of code, we were able to achieve order-of-magnitude speedups over the serial version. Haskell's functional parallelism model is ideal for embarrassingly parallel problems like this one, and promises good scaling to larger instances of the problem over increasing computational resources.

In Section \ref{HORtypeInf} we discuss how type matching and the \codeinline{noRType} tag reduce the number of refinement type inferences we make. Recall that even if both types have a measure (lists and trees), in general there is no guarantee that this is a meaningful comparison. Since \lhask/ is the largest cost to our system in the offline stage, removing refinement type inference in these ambiguous cases provides a large performance gain. As an example, in processing the Haskell standard library \codeinline{base:Prelude}, 7 out of 30 higher order functions do not need to be checked against refinement types using this approach.


\subsection{Example Generation}\label{languageSupport}

We have tried to avoid code analysis at every stage of this paper.
However there are two points where this has fallen short.
First, we must parse a file to extract the name and type information of every top level identifier.
Second, using \lhask/ as a blackbox means that we are limited by \lhask/'s ability to deduce refinement types over functions.
Our eventual goal is to create a system that can be easily ported across functional languages.
Luckily, the first code dependency is small enough to handle with ease in most typed languages (the grammar of a type signature is relatively small). However \lhask/ is a powerful tool that would be difficult to recreate in another language.

To this end, we can extend the refinement type system by allowing refinement type inference on representative examples of a higher order function.
We do not need to identify a particular component function since we are only interested in size based refinement types.
We then apply a similar refinement type inference strategy as in Listing \ref{exRTypeGen} to these examples.

Our current example generation tool uses QuickCheck to generate and apply many examples for higher order function Haskell~\cite{quickcheck}.
Of course, since \lhask/ supports so much of Haskell, this is not practically useful for us, but provides a prototype as a proof of concept.
Imagining that we could not find a refinement type directly on map, we might use examples to infer a refinement type. Take the following code:

\begin{lstlisting}[numbers=none]
map :: (a -> b) -> [a] -> [b]
map f [] = []
map f x:xs = f x : map f xs

mapExs = [[1,2,3] :-> [4,2,8]]
\end{lstlisting}

There are however repercussions to this approach. We are not guaranteed to generate a correct refinement type because we might not generate a fully representative examples. It then seems it is possible prune away many high order functions that are actually useful, but the full repercussions of this is outside the scope of this paper.


\section{Related Work}
\label{sec:related}

From \cite{Osera:2016} -
  "One of the limitations of our approach is that it is incomplete.
  The system does not automatically generate new “helper functions” auxiliary recursive functions that take additional or different arguments.
  It also does not infer instantiation of polymorphic library functions, and the theoretical framework restricts the combination of union and intersection types"
In contrast, our approach supports not only polymorphic first-order library functions, but also higher order functions from libraries - a key part of the functional programming experience. 

In describing \ourTool/, we have shown how relying on a type-directed synthesis approach frees us from burdensome constraints of code analysis and hard coded inference rules, and allows \ourTool/ to synthesis natural and organic code. Many of the techniques we have used have been explored in various contexts before, though generally for the purpose of lower level synthesis. In this section we make some comparisons to related work, and highlight the differences we employ that help us generate readable code.

One of the most closely related works in aspirations is MagicHaskeller~\cite{DBLP:conf/aaip/Katayama09}. This project makes heavy use of a ranking system based on code use and lookup frequency in a database to deliver natural results to the user. In contrast to our work, MagicHaskeller uses a database of functions as its main synthesis engine, with the current database hovering around
64GB~\cite{DBLP:conf/agi/Katayama15}. From this work, we take the inspiration of supporting imported libraries for creating natural code. However, it is important that the system is more portable and easily manipulated by the user - in particular by allowing user defined function in synthesis.

MagicHaskeller work is in the same AI focused domain of inductive programming as the tool IgorII~\cite{DBLP:conf/aaip/HofmannKS09}. IgorII however takes a very code analysis heavy approach, having been originally developed for Maude, then ported to Haskell.

One of the motivating works for exploring type-directed programming by example, especially over recursively defined datatypes is MYTH~\cite{Osera:2015, Osera:2016}. The natural extension of this work in the usability direction was to include a more lightweight and flexible support for user defined and imported datatypes. The $\Lambda^2$ tool also focuses on deriving programs over recursively defined datatypes~\cite{Feser:2015}. One of the major barriers to an average user with these tools, is that the generated code operates on the inner workings on a datatype. While this provides a complete picture of all the data manipulation, often a user might prefer to simply be provided with high level, functioning code. Building in support and the ability to reason on user defined functions in \ourTool/ has made this natural synthesis possible.

Another feature these works support is the synthesis of first order function.
According to our problem specification in Section \ref{problem}, we are only focused on synthesizing higher order functions.
One common synthesis problem these works present is to append an item to a list.
If we were to extend our strategy to first order functions, \ourTool/ would just search through the libraries to find the \codeinline{append} function, as the most natural solution.
With the eventual goal of building a complete program synthesis engine, we will need to integrate with more advanced first order function synthesis systems.
While this problem has been investigated in isolation, it is not clear how to efficiently determine if a set of examples more naturally calls for a higher order function or a first order function.

One direction to explore for first order synthesis is the type reduction algorithm in Section \ref{initVals}.
In providing initial values for component functions, we current are strictly reducing the number of kinds in a type signature.
While this gives us a termination of search guarantee and completeness over the search space, that is not particularly useful guarantee in this case.
Imagining \ourTool/ integrated with an IDE, it would be better to keep the tool running constantly to infinitely search for new suggestions to the code.
To this end we could also enable non-reducing types reducing applications, which would create an infinite recursion, but allow us to find many more functions.
Rather than supplying values to functions, we could supply more functions.

\begin{lstlisting}[numbers=none]
-- given a component function
f :: Int -> Int
f = (+1)
-- `apply' a function rather than a value
f' :: Int -> Int
f' = f . (+2)

\end{lstlisting}

The relationship between refinement types and examples is explored in detail in~\cite{Osera:2016}, leading them to use refinement types to extend the specification language of a programming by example system. Instead, \ourTool/ keeps refinement types entirely as a backend logical inference technique and hidden from the user. In line with our goal of synthesizing natural code, we wish to minimize the asking the onerous task of users to learn to write new specifications. Simple as they may be, refinement types are unlikely to seem as approachable as the more familiar ``examples as a specification'' to the average user.

Taking the refinement types as a specification even further,~\cite{dblp1683325} proposes a system Synquid, that will synthesize programs based on refinement types. At first glance this seems to be an entirely different approach than \ourTool/, which intead uses examples as the specification and only uses refinement types as a search space pruning tool. However, the work in~\cite{Osera:2016} does give an indication that our use of refinement types may be related on fundamental, proof theoretic level. By exploring this relationship in more depth, it may be possible to draw a stronger parallel between these various works and port ideas from one system to another.

One of the most widely used and well known instances of a programming by example system being used by many novice users is FlashFill\cite{GulwaniHS12}. Like other system, the goal of this work is to make executable, not code. This leaves users without the ability to modify generated source code. In the case of FlashFill, being embedding within production level software, most users are not clamoring for this feature. However it would certainly open an interesting avenue to introduce new users to computer programming if this were an option. StriSynth~\cite{icse} takes a step in this direction by providing a natural language description of the synthesized program, but the source code remains obfuscated and inaccessible.

One difficult limitation is that without subexample generation we cannot recursively apply our algorithm as in the $\Lambda^2$ tool~\cite{Feser:2015}.
Subexample generation gives the ability to recursively call the synthesis engine to generate programs with multiple applications of high order functions.
However, since the ability of $\Lambda^2$ to generate subexamples relied on hard coded subexample generation hypotheses for the predefined set of higher order functions, this does not scale.
While inferring the hypotheses might be possible by inspecting the code, we have maintained a dedication to minimize our reliance on code analysis techniques for portability and longevity of the system.
The best way to automatically create subexample generation functions solely based on type information remains a difficult problem.


\section{Conclusions}
\label{conclusions}
Even for novice computer users, the need for basic programming skills is increasing. To meet this need, synthesis tools must be able to produce code that is not only correct, but also useful to the users of these tools. Program synthesis is powerful, but much of this power is left untapped when results are obfuscated by level upon level of folds, filters, and maps. Correctness is only a baseline, if program synthesizers do not produce natural code, most users will not be able to take full advantage of this field.

There are many directions left to explore. Enhancing our refinement type inference from examples will allow greater search space pruning for practical efficiency benefits. The offline / online structure of our algorithm allows novice users to interact with the useful synthesis part of our algorithm without dealing with the complicated refinement-type formulation of our search space pruning, so our tool would integrate well into a live programming environment. Finally, the use of a more powerful language like Agda should be considered as an easier way to manipulate the type directed search.

Finally, as we saw in Section \ref{evaluation}, our tool responds to synthesis queries with highly-readable, transparent, and ultimately natural functions that fit the examples provided. We believe our tool is a step toward realizing a robust and accessible program synthesis system.


\bibliographystyle{abbrvnat}
\bibliography{myBib}

\end{document}

\subsection{Component function generation}\label{makeFxns}
% makeFxns

Recalling the solution space of programs defined in Section \ref{problem}, \ourTool/ must now find first order function for each of the higher order functions that are still candidates (line 18 of Listing \ref{listing:Algo}).
For a given higher order function, \ourTool/ can choose component functions by reusing the weighted type matching algorithm from Listing \ref{valueAlgo}.
Since examples must be given as a concrete type, we can always partially specialize our candidate higher order function. 
We then search for first order functions that will type check against the partially specialized component signature.
This partial specialization is a way of extracting more information out of our examples, and significantly reduces the space of candidate first order functions.
Similar to Listing \ref{horank}, we show an example of how type matching is applied over first order functions in Listing \ref{comprank}.

\begin{lstlisting}[caption=Ranking component function,label=comprank]
examples ::            [Int] -> [Int]
map ::   (a   -> b)   -> [a]   -> [b]
mapEx :: (Int -> Int) -> [Int] -> [Int]

component ::
      Int    -> Int
f1 :: a      -> b      -- value is 21
f2 :: Int    -> a      -- value is 31
f3 :: Int    -> Int    -- value is 41
f4 :: [Bool] -> [Bool] -- value is Nothing
\end{lstlisting}

\subsection{Initial Values}\label{initVals}

In addition to finding first order functions where the arity of the kinds is equal to the component function, we may also want ``larger'' functions that have been applied to initial values.
For examples, if the component signature is \codeinline{::Int->Int}, we may have the first order functions \codeinline{(+)::Int->Int->Int} in scope.
By applying some initial values to \codeinline{(+)}, we can get a new function (e.g. \codeinline{(+1)::Int->Int}) that fits the component signature.

If the initial value's type is an instance of Monoid, we can extract the unit value (named mempty in Haskell's monoid typeclass\cite{monoid}) to use as our initial value. For lists, the unit element is []. However, there are two valid monoids for numbers, using either (+) or (*) as the operators and resulting in unit elements 0 and 1 respectively. We take both of these values (along with other common, useful values of -1, and 2) as possibilities since the cost of testing both values is relatively small.

Additionally, requiring our users to write monoid instances for their datatypes may be a nuisance. However, users may have some domain knowledge that a particular value, or set of values, may be useful in their application. Since our system automatically considers functions defined in the user code base, users may simply write their own specializations of the higher order functions, or provide useful initial values, to be used in synthesis. 

\begin{lstlisting}[caption=adding default initial values]
-- to use 5 as an initial value for foldl
foldl :: (a -> b -> a) -> [b] -> a
foldl5 f i o = foldl f 5 i o

-- to use 5 as an initial value in all recursions
x :: Int
x = 5
\end{lstlisting}

Presented with the problem of finding integer values to satisfy the examples may initially seem like a good application for an SMT solver.
However, keep in mind that we do not in general know what we are trying to solve - the actual use of these variables is hidden within the function definition. Since in this work we maintain a primarily type directed approach, rather than code analysis, we will not be able to unravel these functions.

We must also address the issue first presented in Section \ref{problem}, that it is possible for a higher order function to need initial values in addition to a component function.
For example, the \codeinline{map} function only takes a first order function, while \codeinline{foldl :: (a-> b-> a)-> a-> [b]-> a} requires an initial value for \codeinline{a}.
Using a similar process as for first order function application, we can apply values until the higher order function only needs the example input to complete execution.
To identify initial values in a higher order type signature, we can use our previous assumption that all higher order function have been partially curried to the type \codeinline{_ -> *-> *}. 
Adding the further assumption that only one first order function maybe be passed to the higher order function, we simply tag any non-function type in the hole as an initial value.

%In the implementation, we actually build new functions with the name of the composed functions, and adjust the type signature accordingly.
